% !TEX program = pdflatex
% !BIB program = biber
\documentclass{article}
\usepackage[utf8]{inputenc}
\usepackage[letterpaper,top=2cm,bottom=2cm,left=3cm,right=3cm,marginparwidth=1.75cm]{geometry}
\usepackage{amsmath, amssymb, amsthm}
\usepackage{algorithm, algpseudocode}
\usepackage{float}
\usepackage{listings}
\usepackage{xcolor}
\usepackage{booktabs}
\usepackage{graphicx}
\usepackage[backend=biber, style=authoryear]{biblatex}
\addbibresource{references.bib}
\usepackage{siunitx}
\usepackage{multirow}
\usepackage{subcaption}
\usepackage[strings]{underscore}
\usepackage{hyperref}
\usepackage{pgfplots}
\usepackage{threeparttable}
\pgfplotsset{compat=1.18}

% ========== Couleurs pour le code ==========
\definecolor{codegreen}{rgb}{0,0.6,0}
\definecolor{codegray}{rgb}{0.5,0.5,0.5}
\definecolor{codepurple}{rgb}{0.58,0,0.82}
\definecolor{backcolour}{rgb}{0.95,0.95,0.92}

\lstdefinestyle{mystyle}{
    backgroundcolor=\color{backcolour},
    commentstyle=\color{codegreen},
    keywordstyle=\color{magenta},
    numberstyle=\tiny\color{codegray},
    stringstyle=\color{codepurple},
    basicstyle=\ttfamily\footnotesize,
    breakatwhitespace=false,
    breaklines=true,
    captionpos=b,
    keepspaces=true,
    numbers=left,
    numbersep=5pt,
    showspaces=false,
    showstringspaces=false,
    showtabs=false,
    tabsize=2
}
\lstset{style=mystyle}

\title{Hierarchical Regime Detection and Contagion Analysis in Limit Order Books: A Temporal Wasserstein Framework with Transfer Entropy}
\author{Alexis Noir-Luhalwe}
\date{\today}

\begin{document}

\maketitle

% ========== ABSTRACT ==========
\begin{abstract}
I develop a hierarchical framework for detecting regime-dependent contagion in high-frequency limit order books by combining robust MAD normalization, temporal Wasserstein distances, and a two-level Hidden Markov Model architecture. At the first level, per-ticker HMMs with individually optimized parameters extract local regime probabilities from temporal distributional ruptures in microstructure metrics (micro-price returns, OBI, OFI). At the second level, a Meta-HMM aggregates these local state probabilities to identify sector-wide regimes, while a parallel Direct Global HMM trained on concatenated Wasserstein features provides a benchmark. Using LOBSTER data for five technology stocks on June 21, 2012, I detect heterogeneous local regime structures across tickers and weak but statistically significant global synchronization. Transfer Entropy analysis reveals directed causal relationships, identifying a ``Patient Zero'' of contagion propagation. Lead-lag analysis documents regime-dependent predictability, with OFI exhibiting persistent clustering patterns. An event study of a GOOG OFI spike confirms cross-sectional reallocation dynamics consistent with active portfolio rebalancing. Robustness validation through ARI comparisons, Maximum Mean Discrepancy diagnostics, and meta-vs-direct model concordance confirms that identified regimes represent genuine market structures rather than parametric artifacts.
\end{abstract}

\textbf{Keywords:} Limit Order Book, Hierarchical Hidden Markov Model, Transfer Entropy, Wasserstein Distance, Regime Detection, Contagion, Market Microstructure, MAD Normalization

% ========== INTRODUCTION ==========
\section{Introduction}
\label{sec:intro}

The increasing fragmentation and algorithmic complexity of modern financial markets have transformed the dynamics of liquidity provision and price discovery. In this context, the study of limit order book (LOB) microstructure has emerged as a critical field for understanding contagion and regime-dependent stress propagation across assets. A fundamental challenge is measuring distributional regime shifts and their propagation in a way that captures the full richness of non-linear dependencies that define modern, fragmented markets. Traditional correlation-based measures are inadequate for this task: they capture only linear, first-moment relationships and are blind to changes in tails, asymmetry, and higher-order structure that dominate during stress episodes.

This paper introduces a temporal Wasserstein framework for regime detection and contagion analysis in high-frequency LOB data. Unlike correlation matrices, Wasserstein distance compares entire probability distributions, making it sensitive to non-linear dependencies and distributional ruptures that correlations overlook. I adapt this idea to a \textit{temporal} setting by comparing each metric's distribution \textit{before} and \textit{after} each time point, producing a high-frequency stress signal that spikes precisely at regime transitions. To ensure robustness in noisy microstructure data, I apply Median Absolute Deviation (MAD) normalization, which is non-parametric and resilient to outliers without imposing distributional assumptions.

Traditional approaches to regime detection in LOB data suffer from three interrelated limitations. Univariate HMM models \parencite{slupinski2020hidden, wisebourt2011hierarchical, kth2017online} capture latent states within individual assets but overlook spatial interdependencies across assets. Multi-asset LOB models \parencite{kercheval2024attention, zhou2025tlob} introduce advanced architectures to capture spatial dependencies yet rely on correlation-based measures that assume static, linear relationships---assumptions invalidated by regime-dependent, non-linear stress propagation. Finally, standard parametric filtering at high frequency can be unstable in the presence of outliers and irregular distributions, blurring genuine regime changes with filtering artifacts.

My framework addresses these limitations through four methodological innovations. First, I apply robust MAD normalization to extract stable innovations without parametric assumptions. Second, I compute \textit{temporal} Wasserstein distances that compare pre- and post-distributions of each metric, directly capturing distributional ruptures rather than cross-sectional divergence. Third, I construct a two-level HMM architecture: local HMMs with per-ticker optimized parameters extract state probabilities, which a Meta-HMM aggregates to detect sector-wide regimes, with a parallel Direct Global HMM trained on concatenated features as benchmark. Fourth, I introduce Transfer Entropy to measure directed causal relationships and identify a ``Patient Zero'' of contagion propagation.

I apply the framework to LOBSTER data for five technology stocks (AAPL, INTC, GOOG, AMZN, MSFT) on June 21, 2012. The results reveal heterogeneous local regime structures across tickers, with per-ticker optimization yielding different MAD windows (50--100), Wasserstein windows (50--150), and persistence levels (0.85--0.95). Global synchronization is weak but statistically significant, consistent with moderate intra-sector dependence on a non-crisis day. Transfer Entropy identifies directed causal relationships, with the AAPL--INTC pair exhibiting the strongest bidirectional flow. Lead-lag analysis documents regime-dependent predictability, and an event study of a GOOG OFI spike demonstrates coordinated cross-sectional reallocation. Robustness validation through ARI comparisons, MMD diagnostics, and meta-vs-direct concordance ($\approx$72\% state agreement) confirms the framework's structural validity.

The remainder of this paper is structured as follows. Section~\ref{sec:lit} reviews the literature. Section~\ref{sec:method} details the methodology. Section~\ref{sec:results} presents empirical results. Section~\ref{sec:discuss} discusses implications, and Section~\ref{sec:conclude} concludes.

% ========== LITERATURE REVIEW ==========
\section{Literature Review}
\label{sec:lit}

The analysis of high-frequency market microstructure has undergone a profound transformation with the advent of electronic trading and the fragmentation of liquidity across multiple venues. The literature relevant to this paper can be organized along four complementary axes: univariate LOB modeling, multi-asset contagion analysis, regime-switching frameworks, and the application of Wasserstein distance and information-theoretic measures to financial time series.

The foundational work on univariate LOB modeling using Hidden Markov Models established the groundwork for understanding latent market states. \textcite{slupinski2020hidden} demonstrated how HMMs could effectively model liquidity regimes in individual LOBs by capturing the hidden states driving order book imbalances and queue dynamics. \textcite{wisebourt2011hierarchical} introduced hierarchical HMM structures to predict returns based on LOB features, though still confined to single-asset analysis, and more recently \textcite{kth2017online} implemented non-linear HMM variants for real-time microstructure prediction. While these studies successfully captured the temporal evolution of individual LOBs, they fundamentally overlooked the spatial interdependencies between assets that emerge in modern fragmented markets.

The emergence of multi-asset LOB analysis addressed some of these limitations. \textcite{kercheval2024attention} introduced attention-based networks to model cross-asset relationships in LOB data, using graph structures to represent the complex web of interactions between different stocks' order books. However, this approach relied on correlation-based similarity measures that only capture linear, first-moment relationships. \textcite{zhou2025tlob} further developed this line with transformer-based architectures, explicitly acknowledging the non-stationarity of spatial correlations as a major limitation. These correlation-based approaches contrast sharply with the distributional regime-switching behavior observed in real markets, where stress propagation manifests not merely as increased correlation but as fundamental shifts in the joint distribution of order flow and liquidity metrics.

The regime-switching literature provides the theoretical foundation for understanding these dynamic patterns. Pioneering work by \textcite{hamilton1989new} established Markov-switching models for business cycle analysis, while adaptations by \textcite{agnolucci2019market} and \textcite{zheng2020regime} applied these techniques to financial time series. However, these studies typically focused on aggregated return or volatility measures rather than granular LOB dynamics, and relied on moment-based statistics to characterize regimes, thereby overlooking the full distributional information.

The Wasserstein distance from optimal transport theory \parencite{villani2009optimal, peyre2019computational} quantifies the minimal cost of transforming one probability distribution into another, providing a geometrically interpretable measure of distributional divergence. \textcite{peyre2019computational} provides a comprehensive treatment of computational aspects, while \textcite{fournie2019wasserstein} establishes theoretical foundations for applying Wasserstein distance to financial time series. \textcite{horvath2021clustering} demonstrate the utility of Wasserstein distance for clustering market regimes, showing that Wasserstein-based detection captures distributional shifts invisible to correlation matrices. However, their framework operates at daily frequency, applies K-means without temporal structure, and focuses on univariate return series. My framework addresses these limitations through sub-second temporal resolution, robust MAD normalization, temporally persistent HMM estimation, and a hierarchical cross-asset architecture that captures sector-wide regime coordination.

Transfer Entropy \parencite{schreiber2000measuring} extends mutual information to measure directed information flow between time series, providing a non-parametric measure of causal influence. In finance, Transfer Entropy has been applied to identify information leaders in equity markets and to detect contagion pathways during crises. My framework combines Transfer Entropy with regime-based state probabilities rather than raw returns, enabling detection of causal relationships in regime dynamics---who initiates regime transitions rather than who moves prices first.

% ========== METHODOLOGY ==========
\section{Methodology}
\label{sec:method}

My analytical framework integrates robust normalization, temporal Wasserstein distance computation, a hierarchical HMM architecture, and information-theoretic contagion metrics. The pipeline proceeds in five stages: data loading and LOB metric construction, MAD normalization, temporal Wasserstein feature extraction, hierarchical HMM estimation with per-ticker optimization, and contagion analysis via Transfer Entropy and lead-lag dynamics.

\subsection{Data and LOB Metrics}

I utilize LOBSTER (Limit Order Book System -- The Efficient Reconstructor) data \parencite{lobster2019} for five technology stocks (AAPL, INTC, GOOG, AMZN, MSFT) on June 21, 2012 (9:30--16:00 ET). For each asset, I process both order book snapshots (top 5 levels) and individual message events to reconstruct the complete LOB evolution with millisecond precision. Data are synchronized to a common 500ms grid using forward-fill interpolation, yielding approximately 46,400 observations per asset.

I compute three metrics capturing complementary aspects of market dynamics. The micro-price serves as the primary measure of fair value, calculated as a volume-weighted average of the bid and ask prices across the first $n=5$ levels:
\begin{equation}
M_{t,n} = \frac{\sum_{i=1}^{n} (P_{i,t}^{bid} \cdot Q_{i,t}^{ask} + P_{i,t}^{ask} \cdot Q_{i,t}^{bid})}{\sum_{i=1}^{n} (Q_{i,t}^{bid} + Q_{i,t}^{ask})}
\end{equation}
I use log-returns of the micro-price ($r_t = \log M_t - \log M_{t-1}$) scaled by 100 as the price metric. The Order Book Imbalance (OBI) measures the static relative supply-demand pressure:
\begin{equation}
\text{OBI}_{t,n} = \frac{\sum_{i=1}^{n} Q_{i,t}^{bid} - \sum_{i=1}^{n} Q_{i,t}^{ask}}{\sum_{i=1}^{n} Q_{i,t}^{bid} + \sum_{i=1}^{n} Q_{i,t}^{ask}}
\end{equation}
The Multi-Level Order Flow Imbalance (OFI) tracks the dynamic net liquidity flow by aggregating changes in sizes across the first $n=5$ levels:
\begin{equation}
\text{OFI}_{t,n} = \sum_{i=1}^{n} \left( e_{i,t}^{bid} - e_{i,t}^{ask} \right)
\end{equation}
where for each level $i$, the bid-side contribution $e_{i,t}^{bid}$ accounts for price improvements, size changes at constant price, or price retreats. This multi-level approach, extending the framework of \textcite{cont2014price}, provides a more stable representation of net flow by internalizing re-quotes within the top 5 levels of the book.

\subsection{Robust MAD Normalization}
\label{sec:mad}

A critical departure from standard parametric filtering is the adoption of Median Absolute Deviation (MAD) normalization. High-frequency microstructure data exhibit extreme outliers---flash quotes, quote stuffing, transient liquidity vacuums---that can destabilize parametric volatility models and iterative estimators.

MAD normalization provides a robust, non-parametric alternative. For a rolling window of size $w$, I compute the robust z-score:
\begin{equation}
Z_t = \frac{X_t - \widetilde{X}_t}{1.4826 \cdot \text{MAD}_t + \epsilon}
\end{equation}
where $\widetilde{X}_t = \text{median}(X_{t-w}, \ldots, X_t)$ is the rolling median, $\text{MAD}_t = \text{median}(|X_{t-w} - \widetilde{X}_t|, \ldots, |X_t - \widetilde{X}_t|)$ is the rolling median absolute deviation, and $\epsilon = 10^{-9}$ prevents division by zero. The constant $1.4826$ ensures consistency with the standard deviation under normality: for a Gaussian distribution, $\text{MAD} \times 1.4826 = \sigma$.

This approach offers three advantages for high-frequency LOB data. The median is a robust estimator with breakdown point 0.5, meaning up to 50\% of observations can be outliers without affecting the estimate. MAD requires no iterative optimization---it is computed in a single pass---eliminating convergence failures common at sub-second frequencies. Finally, MAD makes no distributional assumption on the innovation process, avoiding misspecification bias in conditional volatility estimates. The window size $w$ is optimized per ticker (Section~\ref{sec:optimization}), with values ranging from 50 to 150 observations (25--75 seconds at 500ms frequency), allowing each asset to have its own normalization scale.

\subsection{Temporal Wasserstein Distance}
\label{sec:temporal_wass}

A second key innovation is the shift from \textit{cross-sectional} to \textit{temporal} Wasserstein distances. Rather than computing pairwise distances between assets within a rolling window---which conflates within-asset regime shifts with cross-asset differences---the temporal approach compares the distribution of a single metric \textit{before} and \textit{after} each time point. This design yields a direct signal for distributional ruptures, aligned with the timing of regime transitions.

Formally, for asset $i$ and metric $m$, I compute a temporal 1-Wasserstein distance between two adjacent windows centered at time $t$:
\begin{equation}
W_t^{(i,m)} = W_1\!\left(\{Z_{t-w}^{(i,m)}, \ldots, Z_{t}^{(i,m)}\},\, \{Z_{t}^{(i,m)}, \ldots, Z_{t+w}^{(i,m)}\}\right)
\label{eq:temporal_wass}
\end{equation}
where $W_1$ denotes the 1-Wasserstein distance, $Z^{(i,m)}$ is the MAD-normalized series for asset $i$ and metric $m$, and $w$ is the window size. This produces a temporal stress series for each ticker--metric pair, which becomes the observation input for the HMM.

The temporal formulation presents three advantages. First, it detects \textit{when} a distributional change occurs within each asset, rather than \textit{how different} two assets are---a more direct signal for regime transitions. Second, each $W_t^{(i,m)}$ is self-contained per ticker and metric, enabling per-ticker HMM fitting without requiring cross-asset alignment. Third, the temporal Wasserstein series naturally spike during genuine regime transitions, providing a high signal-to-noise input for HMM estimation.

I use the 1-Wasserstein metric because it captures marginal distributional differences without imposing higher-order structure, which is appropriate when temporal structure is modeled downstream by the HMM. The 1-Wasserstein distance admits a quantile formulation:
\begin{equation}
W_1(P, Q) = \int_0^1 \left|F_P^{-1}(u) - F_Q^{-1}(u)\right| \, du
\label{eq:wasserstein_quantile}
\end{equation}
This expression enables computation via sorted quantiles in $O(n \log n)$ time per window. I implement this using \texttt{scipy.stats.wasserstein\_distance}, with optional Numba JIT acceleration that produces numerically identical results (relative difference $< 10^{-4}$) at substantially reduced computation time.

\subsection{Hierarchical HMM Architecture}
\label{sec:hierarchical}

The core architectural innovation is a two-level HMM that resolves three fundamental problems of flat (single-level) HMMs applied to multi-asset LOB data: label switching, noise filtering, and contagion detection.

At the first level, I fit a Gaussian HMM to the temporal Wasserstein features for each ticker $i$ independently. The observation model is $\mathbf{W}_t^{(i)} | S_t^{(i)} = k \sim \mathcal{N}(\boldsymbol{\mu}_k^{(i)}, \boldsymbol{\Sigma}_k^{(i)})$, where $S_t^{(i)} \in \{0, 1, \ldots, K-1\}$ is the latent regime and $\mathbf{W}_t^{(i)} = (W_t^{(i,\text{Price})}, W_t^{(i,\text{OFI})}, W_t^{(i,\text{OBI})})$ are the three temporal Wasserstein features. The covariance structure is set to diagonal by default, reducing complexity from $O(p^2)$ to $O(p)$ parameters, though I implement an adaptive selection: if the maximum absolute correlation between features exceeds $\tau = 0.70$, I switch to full covariance to avoid model misspecification. After EM estimation with 1000 iterations, I enforce temporal persistence by setting the transition matrix to
\begin{equation}
\mathbf{A}_{jk}^{(i)} = \begin{cases}
\pi^{(i)} & \text{if } j = k \\
\frac{1 - \pi^{(i)}}{K-1} & \text{if } j \neq k
\end{cases}
\end{equation}
where $\pi^{(i)}$ is the per-ticker persistence parameter, and post-estimation smoothing via a majority filter of window $s^{(i)}$ eliminates regime assignments shorter than economically meaningful durations. Critically, in addition to the Viterbi-decoded state labels, I extract the posterior state probabilities via the forward-backward algorithm:
\begin{equation}
\mathbf{p}_t^{(i)} = \left(P(S_t^{(i)} = 0 \mid \mathbf{W}_{1:T}^{(i)}), \ldots, P(S_t^{(i)} = K-1 \mid \mathbf{W}_{1:T}^{(i)})\right)
\end{equation}
These continuous probabilities, rather than discrete labels, serve as input to the Meta-HMM, preserving uncertainty information and resolving label switching.

At the second level, the Meta-HMM observes the concatenated local state probabilities across all $N$ tickers: $\mathbf{X}_t^{\text{meta}} = [\mathbf{p}_t^{(1)}, \ldots, \mathbf{p}_t^{(N)}] \in \mathbb{R}^{N \times K}$. This feature vector captures the full probabilistic state of the sector at each time $t$. The Meta-HMM is itself a Gaussian HMM with $K_G$ global regimes, applying the same persistence and smoothing procedure at the global level with potentially stronger persistence to reflect the greater inertia of sector-wide states. This hierarchical architecture resolves label switching implicitly---the Meta-HMM learns the semantic mapping by identifying global patterns in the joint probability space regardless of local label assignments---filters noise through a consensus mechanism that only registers global transitions when multiple tickers exhibit coordinated changes, and directly detects contagion as coordinated probability shifts across assets.

As a benchmark, I also fit a Direct Global HMM on the concatenated temporal Wasserstein features across all tickers: $\mathbf{X}_t^{\text{direct}} = [\mathbf{W}_t^{(1)}, \ldots, \mathbf{W}_t^{(N)}] \in \mathbb{R}^{N \times 3}$. This model bypasses the local HMM layer entirely, fitting directly on $N \times 3$-dimensional features. Comparing the two global models reveals whether the two-level hierarchy adds value beyond a flat global approach.

\subsection{Per-Ticker Parameter Optimization}
\label{sec:optimization}

A distinctive feature of my framework is systematic per-ticker parameter optimization. Rather than imposing a single set of hyperparameters across all assets, I search over a grid of configurations for each ticker independently, varying the MAD window ($w_{\text{MAD}} \in \{50, 100, 150\}$), the Wasserstein window ($w_{\text{Wass}} \in \{50, 100, 150\}$), the local persistence ($\pi \in \{0.85, 0.90, 0.95\}$), the smoothing window ($s \in \{10, 20, 30\}$), and the number of regimes (fixed at $K = 3$). For each configuration, I evaluate using a composite score:
\begin{equation}
\text{Score} = \text{ARI}(\text{HMM}, \text{K-means}) - \lambda \cdot \text{MMD}_{\text{penalty}}
\end{equation}
where the Adjusted Rand Index (ARI) measures agreement between HMM regimes and K-means clusters---ensuring the HMM captures genuine structure detectable by non-parametric methods---and the MMD penalty discourages configurations where Maximum Mean Discrepancy between regimes is inconsistent with the expected metric separation. The weight $\lambda = 0.10$ balances exploration and exploitation. This optimization is parallelized across tickers and configurations using \texttt{concurrent.futures}.

\subsection{Contagion Metrics}
\label{sec:contagion}

Transfer Entropy (TE) measures the directed information flow from a source asset to a target asset, quantifying how much knowing the source's past reduces uncertainty about the target's future beyond what the target's own history provides:
\begin{equation}
\text{TE}(X \rightarrow Y) = \sum_{y_{t+1}, y_t, x_t} p(y_{t+1}, y_t, x_t) \log \frac{p(y_{t+1} | y_t, x_t)}{p(y_{t+1} | y_t)}
\end{equation}
where $x_t$ and $y_t$ are discretized versions of the source and target stress probabilities (sum of non-calm regime probabilities). I use $k=2$ lags (1 second) and 10 bins for discretization. The asymmetry $\text{TE}(X \rightarrow Y) - \text{TE}(Y \rightarrow X)$ reveals the net direction of information flow, and I compute a full $N \times N$ TE matrix across all ticker pairs to map the causal network of contagion.

To identify the ``Patient Zero''---the asset that initiates contagion---I combine two complementary metrics: the mean outgoing TE from ticker $i$ to all other tickers, measuring how much causal influence $i$ exerts on the sector, and the leadership score from the Meta-HMM synchronization analysis, measuring how often ticker $i$'s local regime transitions precede global transitions versus follow them. A composite contagion score normalizes both metrics and ranks tickers by their potential to initiate sector-wide regime shifts.

I also measure synchronization between local and global regimes through co-transition analysis: the fraction of global regime transitions that coincide with local transitions for each ticker within a $\pm$5 second window. High synchronization indicates that a ticker's local dynamics are tightly coupled with sector-wide dynamics, while low synchronization suggests idiosyncratic behavior.

\subsection{Robustness Framework}

I validate my framework through four complementary diagnostics. For both local and global models, I compare HMM regime assignments against non-parametric K-means clustering via the Adjusted Rand Index, verifying that identified regimes reflect genuine data structure rather than parametric artifacts. I compute the Maximum Mean Discrepancy (with RBF kernel) between regime-conditional distributions for each metric, verifying that regimes capture genuinely different distributional states. I compare Meta-HMM and Direct Global HMM state assignments to assess whether the hierarchical architecture and the flat global model detect similar structures. Finally, I examine the entropy of global state probabilities to assess whether models produce confident or diffuse regime assignments.


% ========== RESULTS ==========
\section{Results}
\label{sec:results}

\subsection{Per-Ticker Optimization and Local Regimes}

Table~\ref{tab:local_hmm_params} presents the optimized parameters for each ticker, revealing substantial heterogeneity across assets. AAPL and AMZN favor short MAD and Wasserstein windows (50 observations each), with moderate persistence ($\pi = 0.90$ and $0.95$ respectively), suggesting their microstructure dynamics evolve rapidly enough to be captured at relatively local scales. In contrast, GOOG and MSFT require larger windows (MAD $= 100$, Wasserstein $= 150$) and maximal persistence ($\pi = 0.95$), indicating that distributional ruptures in these stocks are more gradual and regime states more persistent. INTC occupies an intermediate position with a short MAD window (50) but longer Wasserstein window (100) and the lowest persistence ($\pi = 0.85$), reflecting a microstructure characterized by rapid local volatility changes but slower distributional shifts. This heterogeneity is economically meaningful: it reflects differences in market-making activity, institutional ownership concentration, and algorithmic trading intensity across stocks.

\input{tables/local_hmm_params.tex}

The resulting local regime distributions, presented in Table~\ref{tab:local_regime_distribution}, confirm that regime structure is fundamentally asset-specific. GOOG spends 95.1\% of the trading day in a single dominant regime with only brief excursions into stress states, consistent with its relatively low intraday volatility on this date. INTC exhibits a similarly concentrated structure (80.6\% in the dominant regime) but with a more substantial secondary regime (13.3\%). At the other extreme, MSFT distributes its time across regimes more evenly (7.5\% / 29.3\% / 63.2\%), suggesting richer microstructure dynamics with more frequent regime transitions. AAPL falls between these extremes (56.9\% / 36.6\% / 6.5\%), while AMZN mirrors INTC's concentrated structure with a different secondary allocation.

\input{tables/local_regime_distribution.tex}

Figure~\ref{fig:local_regime_characteristics} illustrates the microstructure characteristics of local regimes for AAPL, showing how the three regimes are differentiated by their temporal Wasserstein feature profiles. The clear separation across regimes---particularly in the OFI dimension---validates that the HMM captures genuinely distinct microstructure states rather than arbitrary partitions of a continuous feature space.

\begin{figure}[H]
\centering
\includegraphics[width=0.95\textwidth]{figures/regime_characteristics_local_AAPL.png}
\caption{Local regime characteristics for AAPL. Error bars represent $\pm 1$ standard deviation. The separation across regimes, particularly for OFI, validates that the HMM captures genuinely distinct microstructure states. Comparable figures for the remaining four tickers are presented in Appendix~\ref{app:local_chars}.}
\label{fig:local_regime_characteristics}
\end{figure}

\subsection{Global Regimes and Synchronization}

The Meta-HMM and Direct Global HMM produce global regime assignments with approximately 72.1\% state agreement (Table~\ref{tab:sync_global_comparison}), indicating that both approaches detect broadly similar sector-wide patterns despite fundamentally different input representations---local state probabilities for the Meta-HMM versus raw Wasserstein features for the Direct model. This substantial but imperfect concordance suggests that the two models capture complementary aspects of sector dynamics. The entropy analysis (Table~\ref{tab:entropy_global}) reveals a striking contrast in confidence levels: the Meta-HMM produces near-deterministic regime assignments (mean entropy $= 0.000$), while the Direct model exhibits slightly higher uncertainty (entropy $= 0.002$). This difference arises because the Meta-HMM operates on already-filtered local state probabilities, which compress the information and reduce ambiguity, whereas the Direct model must extract regime structure directly from raw Wasserstein features.

\input{tables/sync_global_comparison.tex}

\input{tables/entropy_global.tex}

Table~\ref{tab:local_global_sync} presents the synchronization metrics between local and global regimes, revealing the coupling strength between each ticker and the sector-wide dynamics. GOOG exhibits the highest synchronization rate (22.0\%) with 13 co-transitions out of 59 global transitions, but paradoxically shows a leadership score of $-1.0$, indicating that it consistently \textit{lags} rather than leads global transitions. AMZN shows a moderate synchronization rate (8.5\%) with a positive leadership score ($+0.09$), suggesting it slightly anticipates sector-wide regime shifts. INTC and AAPL display zero synchronization despite having nonzero leadership scores, indicating that their local transitions occur frequently enough but never coincide with global transition windows. These weak synchronization rates are consistent with moderate intra-sector dependence on a non-crisis day---a baseline against which crisis-day synchronization can be measured.

\input{tables/local_global_sync.tex}

Figure~\ref{fig:temporal_comparison_aapl} compares the local, meta-global, and direct-global regime timelines for AAPL, illustrating the relationship between the three levels of regime detection. The partial alignment between local and global timelines reflects the moderate coupling between AAPL's microstructure dynamics and sector-wide patterns, while the discrepancies highlight periods where AAPL's idiosyncratic dynamics diverge from the sector.

\begin{figure}[H]
\centering
\includegraphics[width=0.95\textwidth]{figures/temporal_regime_comparison_AAPL.png}
\caption{Temporal regime comparison for AAPL. Top: local HMM regimes. Middle: Meta-HMM global regimes. Bottom: Direct Global HMM regimes. Comparable figures for all tickers are in Appendix~\ref{app:temporal_comparisons}.}
\label{fig:temporal_comparison_aapl}
\end{figure}

\subsection{Stress Decomposition and Lead-Lag Dynamics}

Figure~\ref{fig:stress_decomp_meta} presents the ticker-specific stress decomposition with Meta-HMM regime overlays. The association between regime transitions and stress spikes validates the regime identification: periods classified as high-stress regimes by the Meta-HMM correspond to visible elevations in the temporal Wasserstein distances across multiple tickers simultaneously, particularly in the OFI dimension.

\begin{figure}[H]
\centering
\includegraphics[width=\textwidth]{figures/stress_decomposition_meta.png}
\caption{Stress decomposition by ticker and metric with Meta-HMM regime overlay. Top: Price stress. Middle: OBI stress. Bottom: OFI stress. Colored backgrounds indicate Meta-HMM global regimes.}
\label{fig:stress_decomp_meta}
\end{figure}

The lead-lag analysis between local ticker stress and global sector stress (Table~\ref{tab:leadlag_local_global}) reveals which tickers anticipate or follow sector-wide dynamics. INTC exhibits the strongest leading behavior with a positive alpha score of $0.011$ and peak correlation at $+10$ seconds lag, indicating that INTC's local Wasserstein stress systematically precedes global stress. GOOG also leads modestly (alpha $= 0.005$), while AAPL and AMZN lag global dynamics (negative alpha scores of $-0.008$ and $-0.032$ respectively). MSFT occupies a near-neutral position (alpha $= 0.001$) with peak correlation at $+4$ seconds, suggesting it tracks global dynamics almost contemporaneously.

\input{tables/leadlag_local_global_top.tex}

The cross-ticker lead-lag relationships (Table~\ref{tab:leadlag_between_tickers}) identify pairs with significant temporal dependencies. The INTC--MSFT pair exhibits the strongest correlation (0.253 at $+10$ seconds lag), indicating that INTC's temporal Wasserstein stress is a strong predictor of MSFT stress 10 seconds later. The AAPL--AMZN pair shows the second strongest relationship (0.222 at $+7$ seconds), while several other pairs display significant but weaker correlations. Notably, AAPL--GOOG and INTC--AMZN show near-zero correlations, suggesting these pairs exhibit largely independent microstructure dynamics despite belonging to the same sector.

\input{tables/leadlag_between_tickers_top.tex}

Figure~\ref{fig:leadlag_grid} presents the full $3 \times 3$ multi-metric lead-lag analysis across stress quantiles, revealing that OFI exhibits strong autocorrelation, indicating persistent order flow clustering, and that lead-lag structure intensifies during high-stress periods (Q90 correlations substantially exceed Q10).

\begin{figure}[H]
\centering
\includegraphics[width=\textwidth]{figures/leadlag_multimetric_grid.png}
\caption{Multi-metric lead-lag analysis by stress quantile. Each panel displays cross-correlations between source and target metrics. Q10 $=$ calm, Q50 $=$ normal, Q90 $=$ stress.}
\label{fig:leadlag_grid}
\end{figure}

\subsection{Transfer Entropy and Patient Zero}

Table~\ref{tab:transfer_entropy} presents the top directed information flow relationships between tickers. The AAPL$\rightarrow$INTC pair exhibits the strongest Transfer Entropy (0.000771 nats), followed by the reverse direction INTC$\rightarrow$AAPL (0.000588 nats). This bidirectional but asymmetric flow indicates that AAPL exerts greater causal influence on INTC than vice versa, consistent with AAPL's larger market capitalization and more active institutional trading. The AMZN$\rightarrow$AAPL connection (0.000498 nats) represents the third strongest relationship, while GOOG appears primarily as a receiver rather than a sender of information, with GOOG$\rightarrow$AAPL (0.000264 nats) being its strongest outgoing link.

\input{tables/transfer_entropy_top.tex}

The Patient Zero analysis combines outgoing Transfer Entropy with the leadership score from synchronization analysis. AAPL emerges as the strongest candidate for contagion initiator: it has the highest total outgoing TE (summing AAPL$\rightarrow$INTC, AAPL$\rightarrow$AMZN, and AAPL$\rightarrow$MSFT) and, despite a negative leadership score in the synchronization analysis, its causal influence on other tickers' regime dynamics is unambiguous. INTC serves as a secondary transmission node, with strong bidirectional connections to AAPL and moderate outgoing TE to the rest of the sector.

\subsection{Event Study: GOOG OFI Spike}

Figure~\ref{fig:event_study} documents a representative liquidity event where GOOG experiences a large OFI spike at $t \approx 15{,}000$ (approximately 11:33 AM), followed by coordinated reallocation across the remaining four stocks. The 33-minute context panel reveals the spike's position within the broader intraday dynamics, while the 4-minute zoom shows the synchronized cross-ticker OFI responses with remarkable clarity. GOOG exhibits a large negative OFI change, while MSFT, AMZN, AAPL, and INTC all show positive reallocations, suggesting coordinated portfolio rebalancing rather than passive spillover. This episode demonstrates active capital redistribution: when GOOG experiences liquidity stress, institutional traders redirect capital across the sector following a hierarchical absorption pattern consistent with optimal execution strategies that target the most liquid alternatives.

\begin{figure}[H]
\centering
\includegraphics[width=\textwidth]{figures/event_study_goog_spike_optimal.png}
\caption{Event study: GOOG OFI spike and cross-ticker reallocation. Top: 33-minute context showing the spike at $t \approx 15{,}000$. Bottom: 4-minute zoom revealing synchronized cross-ticker OFI responses.}
\label{fig:event_study}
\end{figure}

\subsection{Robustness Diagnostics}

The ARI diagnostics for global models (Table~\ref{tab:ari_global}) reveal that the Meta-HMM shows higher agreement with K-means (ARI $= 0.174$) than the Direct model (ARI $= 0.129$), suggesting the hierarchical approach captures more of the non-parametric cluster structure. The meta-vs-direct ARI of 0.219 indicates fair agreement between the two global approaches, confirming that they detect related but not identical structures. At the local level (Table~\ref{tab:ari_meta_local}), GOOG shows the highest alignment between its local HMM and the Meta-HMM global model (ARI $= 0.463$), consistent with its high synchronization rate. INTC and AMZN show moderate alignment (ARI $\approx 0.28$), while AAPL (0.073) and MSFT (0.046) exhibit largely independent dynamics relative to the global model.

\input{tables/ari_global_comparisons.tex}

\input{tables/ari_meta_vs_local.tex}

The Maximum Mean Discrepancy between regimes (Tables~\ref{tab:mmd_local} and~\ref{tab:mmd_global}) confirms that regimes capture genuinely distinct distributional states. At the local level, OFI tends to show the strongest regime separation (e.g., GOOG OFI MMD $= 1.436$), consistent with OFI being the primary driver of regime differentiation. At the global level, the Meta-HMM provides stronger OBI separation (0.521 vs. 0.292 for the Direct model), while the Direct model provides stronger OFI separation (0.746 vs. 0.710), suggesting the two models characterize regimes through complementary distributional features.

\input{tables/mmd_local.tex}

\input{tables/mmd_global.tex}


% ========== DISCUSSION ==========
\section{Discussion}
\label{sec:discuss}

\subsection{Methodological Contributions}

My analysis makes four methodological contributions to the market microstructure literature. First, I demonstrate that MAD normalization provides a robust, non-parametric alternative to standard parametric filtering for high-frequency LOB data. The rolling median-based approach eliminates sensitivity to outliers without imposing distributional assumptions, avoids convergence failures common with iterative parametric models at sub-second frequencies, and introduces a tunable window parameter that can be optimized per ticker. This represents a practical advance for production trading systems where robustness to extreme market conditions is paramount.

Second, the temporal Wasserstein formulation---comparing before-vs-after distributions rather than cross-sectional pairwise distances---provides a more direct signal for regime transition detection. By measuring distributional ruptures within each ticker independently, the temporal approach yields per-ticker features that naturally spike during genuine regime changes, improving HMM estimation quality.

Third, the hierarchical HMM architecture resolves the label switching problem inherent in multi-asset regime detection, filters noise through a consensus mechanism, and directly detects contagion as coordinated probability shifts across assets. The comparison between the Meta-HMM and the Direct Global HMM reveals that both approaches detect broadly similar structures ($\approx$72\% state agreement), but with complementary regime characterizations: the Meta-HMM provides stronger OBI separation while the Direct model provides stronger OFI separation. This complementarity suggests that neither approach alone captures the full richness of sector-wide regime dynamics, and that combining their signals could improve regime detection accuracy.

Fourth, the integration of Transfer Entropy with regime-based state probabilities enables detection of directed causal relationships in regime dynamics. Unlike traditional lead-lag analysis based on returns or correlations, TE on regime probabilities measures who initiates regime transitions, providing a principled framework for Patient Zero identification.

\subsection{Economic Implications}

The empirical results on June 21, 2012 reveal several important patterns. The heterogeneity of local regime structures across tickers is a key finding: rather than a single regime structure characterizing the technology sector, each asset exhibits its own microstructure signature with different optimal parameters. This heterogeneity reflects differences in market-making activity, institutional ownership, and algorithmic trading intensity. The weak but significant global synchronization is consistent with expectations for a non-crisis day---the moderate ARI values (0.129--0.219) and low synchronization rates indicate that cross-asset regime alignment is not dominant, and idiosyncratic microstructure dynamics prevail. Transfer Entropy reveals asymmetric information flow, with the AAPL--INTC pair exhibiting the strongest bidirectional connection, consistent with the historical relationship between these stocks as technology sector components with overlapping institutional investor bases.

\subsection{Practical Applications}

The GOOG event study confirms that stress-driven reallocation follows a hierarchical absorption pattern, with more liquid alternatives absorbing the bulk of capital flows. This demonstrates active portfolio rebalancing rather than passive contagion---an important distinction for regulatory assessments of systemic risk in high-frequency markets. From a practical standpoint, the framework enables real-time monitoring of cross-sectional stress and provides actionable intelligence through Patient Zero identification for risk managers seeking leading indicators of sector-wide regime shifts.

Several limitations should be noted. My analysis covers a single trading day; extending to multiple days, including crisis periods, would test the framework's ability to detect varying contagion intensities. The parameter optimization uses a fixed grid rather than adaptive or Bayesian search, which could improve parameter selection efficiency. The Transfer Entropy computation relies on histogram-based discretization, introducing sensitivity to bin count; continuous TE estimators (e.g., KSG estimator) could improve robustness. Finally, the framework currently focuses on within-sector contagion; cross-sector analysis would provide a more complete picture of systemic risk.


% ========== CONCLUSION ==========
\section{Conclusion}
\label{sec:conclude}

This paper introduces a hierarchical regime-detection and contagion framework for high-frequency limit order books that combines four methodological innovations: robust MAD normalization, temporal Wasserstein distances, a two-level HMM architecture (local per-ticker and global Meta-HMM), and Transfer Entropy-based causal analysis. The framework addresses fundamental limitations of existing approaches---outlier sensitivity of parametric filtering, label switching in multi-asset HMMs, and the inability of correlation-based measures to detect directed causal relationships in regime dynamics.

Empirically, I identify heterogeneous local regime structures across five technology stocks, with per-ticker parameter optimization revealing that no single configuration adequately captures the diverse microstructure signatures. Global synchronization is weak but statistically significant, consistent with moderate intra-sector dependence on a non-crisis day. Transfer Entropy reveals directed causal relationships and enables Patient Zero identification, pointing to AAPL as the primary contagion initiator. An event study confirms hierarchical capital reallocation during a GOOG liquidity event, demonstrating active portfolio rebalancing rather than passive spillover.

Robustness validation through ARI comparisons, MMD diagnostics, meta-vs-direct concordance ($\approx$72\%), and entropy analysis confirms that identified regimes represent genuine market structures. The framework provides a principled foundation for regime-aware microstructure analysis, applicable to real-time monitoring of cross-sectional stress and portfolio risk management.

\printbibliography

% ========== APPENDIX ==========
\appendix
\section{Appendix}

\subsection{Local Regime Characteristics}
\label{app:local_chars}

\begin{figure}[H]
\centering
\begin{subfigure}[b]{0.48\textwidth}
\includegraphics[width=\textwidth]{figures/regime_characteristics_local_INTC.png}
\caption{INTC}
\end{subfigure}
\hfill
\begin{subfigure}[b]{0.48\textwidth}
\includegraphics[width=\textwidth]{figures/regime_characteristics_local_GOOG.png}
\caption{GOOG}
\end{subfigure}
\\[0.5em]
\begin{subfigure}[b]{0.48\textwidth}
\includegraphics[width=\textwidth]{figures/regime_characteristics_local_AMZN.png}
\caption{AMZN}
\end{subfigure}
\hfill
\begin{subfigure}[b]{0.48\textwidth}
\includegraphics[width=\textwidth]{figures/regime_characteristics_local_MSFT.png}
\caption{MSFT}
\end{subfigure}
\caption{Local regime characteristics for INTC, GOOG, AMZN, and MSFT. Error bars represent $\pm 1$ standard deviation. Each ticker exhibits distinct regime signatures, confirming the value of per-ticker parameter optimization.}
\label{fig:local_regime_chars_appendix}
\end{figure}

\subsection{Local HMM Timelines and Histograms}
\label{app:local_timelines}

\begin{figure}[H]
\centering
\begin{subfigure}[b]{0.48\textwidth}
\includegraphics[width=\textwidth]{figures/hmm_local_AAPL_timeline.png}
\caption{AAPL}
\end{subfigure}
\hfill
\begin{subfigure}[b]{0.48\textwidth}
\includegraphics[width=\textwidth]{figures/hmm_local_INTC_timeline.png}
\caption{INTC}
\end{subfigure}
\\[0.5em]
\begin{subfigure}[b]{0.48\textwidth}
\includegraphics[width=\textwidth]{figures/hmm_local_GOOG_timeline.png}
\caption{GOOG}
\end{subfigure}
\hfill
\begin{subfigure}[b]{0.48\textwidth}
\includegraphics[width=\textwidth]{figures/hmm_local_AMZN_timeline.png}
\caption{AMZN}
\end{subfigure}
\\[0.5em]
\begin{subfigure}[b]{0.48\textwidth}
\includegraphics[width=\textwidth]{figures/hmm_local_MSFT_timeline.png}
\caption{MSFT}
\end{subfigure}
\caption{Local HMM regime timelines for all five tickers throughout the trading day. The heterogeneous temporal patterns confirm that regime dynamics are fundamentally asset-specific.}
\label{fig:local_timelines}
\end{figure}

\begin{figure}[H]
\centering
\begin{subfigure}[b]{0.48\textwidth}
\includegraphics[width=\textwidth]{figures/hmm_local_AAPL_hist.png}
\caption{AAPL}
\end{subfigure}
\hfill
\begin{subfigure}[b]{0.48\textwidth}
\includegraphics[width=\textwidth]{figures/hmm_local_INTC_hist.png}
\caption{INTC}
\end{subfigure}
\\[0.5em]
\begin{subfigure}[b]{0.48\textwidth}
\includegraphics[width=\textwidth]{figures/hmm_local_GOOG_hist.png}
\caption{GOOG}
\end{subfigure}
\hfill
\begin{subfigure}[b]{0.48\textwidth}
\includegraphics[width=\textwidth]{figures/hmm_local_AMZN_hist.png}
\caption{AMZN}
\end{subfigure}
\\[0.5em]
\begin{subfigure}[b]{0.48\textwidth}
\includegraphics[width=\textwidth]{figures/hmm_local_MSFT_hist.png}
\caption{MSFT}
\end{subfigure}
\caption{Local HMM regime histograms for all five tickers.}
\label{fig:local_histograms}
\end{figure}

\subsection{Temporal Regime Comparisons (All Tickers)}
\label{app:temporal_comparisons}

\begin{figure}[H]
\centering
\begin{subfigure}[b]{0.48\textwidth}
\includegraphics[width=\textwidth]{figures/temporal_regime_comparison_INTC.png}
\caption{INTC}
\end{subfigure}
\hfill
\begin{subfigure}[b]{0.48\textwidth}
\includegraphics[width=\textwidth]{figures/temporal_regime_comparison_GOOG.png}
\caption{GOOG}
\end{subfigure}
\\[0.5em]
\begin{subfigure}[b]{0.48\textwidth}
\includegraphics[width=\textwidth]{figures/temporal_regime_comparison_AMZN.png}
\caption{AMZN}
\end{subfigure}
\hfill
\begin{subfigure}[b]{0.48\textwidth}
\includegraphics[width=\textwidth]{figures/temporal_regime_comparison_MSFT.png}
\caption{MSFT}
\end{subfigure}
\caption{Temporal regime comparisons (local vs. Meta-HMM vs. Direct Global) for all tickers except AAPL (shown in Figure~\ref{fig:temporal_comparison_aapl}).}
\label{fig:temporal_comparisons_appendix}
\end{figure}

\subsection{Global HMM Timelines and Characteristics}
\label{app:global}

\begin{figure}[H]
\centering
\begin{subfigure}[b]{0.48\textwidth}
\includegraphics[width=\textwidth]{figures/hmm_meta_timeline.png}
\caption{Meta-HMM timeline}
\end{subfigure}
\hfill
\begin{subfigure}[b]{0.48\textwidth}
\includegraphics[width=\textwidth]{figures/hmm_direct_timeline.png}
\caption{Direct Global HMM timeline}
\end{subfigure}
\\[0.5em]
\begin{subfigure}[b]{0.48\textwidth}
\includegraphics[width=\textwidth]{figures/hmm_meta_hist.png}
\caption{Meta-HMM histogram}
\end{subfigure}
\hfill
\begin{subfigure}[b]{0.48\textwidth}
\includegraphics[width=\textwidth]{figures/hmm_direct_hist.png}
\caption{Direct Global HMM histogram}
\end{subfigure}
\caption{Global HMM regime timelines and histograms for both the Meta-HMM and Direct Global models.}
\label{fig:global_timelines}
\end{figure}

\begin{figure}[H]
\centering
\begin{subfigure}[b]{0.48\textwidth}
\includegraphics[width=\textwidth]{figures/regime_characteristics_meta.png}
\caption{Meta-HMM regime characteristics}
\end{subfigure}
\hfill
\begin{subfigure}[b]{0.48\textwidth}
\includegraphics[width=\textwidth]{figures/regime_characteristics_global_direct.png}
\caption{Direct Global HMM regime characteristics}
\end{subfigure}
\caption{Global regime characteristics for both models. The Meta-HMM features are the concatenated local state probabilities; the Direct model features are the concatenated temporal Wasserstein distances.}
\label{fig:global_regime_chars}
\end{figure}

\subsection{Synchronization and Entropy}
\label{app:sync}

\begin{figure}[H]
\centering
\begin{subfigure}[b]{0.48\textwidth}
\includegraphics[width=\textwidth]{figures/sync_local_global.png}
\caption{Local$\rightarrow$Global synchronization rates}
\end{subfigure}
\hfill
\begin{subfigure}[b]{0.48\textwidth}
\includegraphics[width=\textwidth]{figures/entropy_global.png}
\caption{Global entropy distribution (Meta vs. Direct)}
\end{subfigure}
\caption{Left: synchronization rates between local and global regime transitions for each ticker. Right: entropy distributions comparing regime assignment confidence of the Meta-HMM and Direct Global models.}
\label{fig:sync_entropy}
\end{figure}

\subsection{Stress Decomposition (Direct Global)}
\label{app:stress_direct}

\begin{figure}[H]
\centering
\includegraphics[width=\textwidth]{figures/stress_decomposition_direct.png}
\caption{Stress decomposition with Direct Global HMM overlay. Compare with Figure~\ref{fig:stress_decomp_meta} (Meta-HMM overlay) to observe differences in regime boundary placement.}
\label{fig:stress_decomp_direct}
\end{figure}

\subsection{Local HMM Feature Distributions}
\label{app:local_features}

\begin{figure}[H]
\centering
\begin{subfigure}[b]{0.48\textwidth}
\includegraphics[width=\textwidth]{figures/hmm_local_AAPL_features.png}
\caption{AAPL}
\end{subfigure}
\hfill
\begin{subfigure}[b]{0.48\textwidth}
\includegraphics[width=\textwidth]{figures/hmm_local_INTC_features.png}
\caption{INTC}
\end{subfigure}
\\[0.5em]
\begin{subfigure}[b]{0.48\textwidth}
\includegraphics[width=\textwidth]{figures/hmm_local_GOOG_features.png}
\caption{GOOG}
\end{subfigure}
\hfill
\begin{subfigure}[b]{0.48\textwidth}
\includegraphics[width=\textwidth]{figures/hmm_local_AMZN_features.png}
\caption{AMZN}
\end{subfigure}
\\[0.5em]
\begin{subfigure}[b]{0.48\textwidth}
\includegraphics[width=\textwidth]{figures/hmm_local_MSFT_features.png}
\caption{MSFT}
\end{subfigure}
\caption{Local HMM feature distributions by regime for all tickers. Boxplots show the distribution of temporal Wasserstein features (Price, OFI, OBI) within each regime.}
\label{fig:local_features_appendix}
\end{figure}

\subsection{Global HMM Feature Distributions}
\label{app:global_features}

\begin{figure}[H]
\centering
\begin{subfigure}[b]{0.48\textwidth}
\includegraphics[width=\textwidth]{figures/hmm_meta_features.png}
\caption{Meta-HMM features by regime}
\end{subfigure}
\hfill
\begin{subfigure}[b]{0.48\textwidth}
\includegraphics[width=\textwidth]{figures/hmm_direct_features.png}
\caption{Direct Global HMM features by regime}
\end{subfigure}
\caption{Global HMM feature distributions by regime. The different feature representations lead to complementary regime characterizations.}
\label{fig:global_features_appendix}
\end{figure}

\subsection{Significant Lead-Lag Relationships}
\label{app:leadlag}

\input{tables/leadlag_significant_global_top.tex}

\subsection{Selected Lead-Lag Plots}
\label{app:leadlag_plots}

\begin{figure}[H]
\centering
\begin{subfigure}[b]{0.48\textwidth}
\includegraphics[width=\textwidth]{figures/leadlag_local_AAPL_OFI_OFI_R0.png}
\caption{AAPL: OFI$\rightarrow$OFI (Local, R0)}
\end{subfigure}
\hfill
\begin{subfigure}[b]{0.48\textwidth}
\includegraphics[width=\textwidth]{figures/leadlag_local_GOOG_OFI_OFI_R0.png}
\caption{GOOG: OFI$\rightarrow$OFI (Local, R0)}
\end{subfigure}
\\[0.5em]
\begin{subfigure}[b]{0.48\textwidth}
\includegraphics[width=\textwidth]{figures/leadlag_meta_AAPL_OFI_OFI_R0.png}
\caption{AAPL: OFI$\rightarrow$OFI (Meta, R0)}
\end{subfigure}
\hfill
\begin{subfigure}[b]{0.48\textwidth}
\includegraphics[width=\textwidth]{figures/leadlag_meta_INTC_OBI_OBI_R0.png}
\caption{INTC: OBI$\rightarrow$OBI (Meta, R0)}
\end{subfigure}
\caption{Selected lead-lag plots showing significant autocorrelation patterns. OFI exhibits persistent clustering (panels a--c), while OBI shows distinct regime-dependent behavior (panel d).}
\label{fig:leadlag_selected}
\end{figure}

\subsection{Implementation Details}
\label{app:implementation}

The complete analysis pipeline is implemented in Python using \texttt{polars} (v1.0+) for high-performance data loading via LazyFrame evaluation, \texttt{scipy.stats} for Wasserstein distance computation and statistical tests, \texttt{hmmlearn} (v0.3) for Gaussian HMM estimation with the forward-backward algorithm, \texttt{scikit-learn} (v1.3) for K-means clustering and ARI/Silhouette analysis, \texttt{numba} for optional JIT-compiled Wasserstein acceleration, \texttt{networkx} for contagion network visualization, and \texttt{concurrent.futures} for parallelized per-ticker optimization.

Local HMMs are fitted with 3 states, diagonal covariance (adaptive to full if $|\rho| \geq 0.70$), 1000 EM iterations, and per-ticker optimized persistence and smoothing. The Meta-HMM uses 3 global regimes with diagonal covariance, while the Direct Global HMM operates on the concatenated 15-dimensional Wasserstein feature space. Transfer Entropy is estimated via histogram-based discretization with $k=2$ lags and 10 bins. MMD uses an RBF kernel with median heuristic bandwidth, and K-means employs 10 initializations with StandardScaler preprocessing.

\end{document}
