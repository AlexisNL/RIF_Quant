% !TEX program = pdflatex
% !BIB program = biber
\documentclass{article}
\usepackage[utf8]{inputenc}
\usepackage[letterpaper,top=2cm,bottom=2cm,left=3cm,right=3cm,marginparwidth=1.75cm]{geometry}
\usepackage{amsmath, amssymb, amsthm}
\usepackage{algorithm, algpseudocode}
\usepackage{float}
\usepackage{listings}
\usepackage{xcolor}
\usepackage{booktabs}
\usepackage{graphicx}
\usepackage[backend=biber, style=authoryear]{biblatex}
\addbibresource{references.bib}
\usepackage{siunitx}
\usepackage{multirow}
\usepackage{subcaption}
\usepackage[strings]{underscore}
\usepackage{hyperref}
\usepackage{threeparttable}

% ========== Couleurs pour le code ==========
\definecolor{codegreen}{rgb}{0,0.6,0}
\definecolor{codegray}{rgb}{0.5,0.5,0.5}
\definecolor{codepurple}{rgb}{0.58,0,0.82}
\definecolor{backcolour}{rgb}{0.95,0.95,0.92}

\lstdefinestyle{mystyle}{
    backgroundcolor=\color{backcolour},
    commentstyle=\color{codegreen},
    keywordstyle=\color{magenta},
    numberstyle=\tiny\color{codegray},
    stringstyle=\color{codepurple},
    basicstyle=\ttfamily\footnotesize,
    breakatwhitespace=false,
    breaklines=true,
    captionpos=b,
    keepspaces=true,
    numbers=left,
    numbersep=5pt,
    showspaces=false,
    showstringspaces=false,
    showtabs=false,
    tabsize=2
}
\lstset{style=mystyle}

% ========== Metadata ==========
\title{Distributional Stress in High-Frequency Limit Order Books: Temporal Wasserstein Regimes and Contagion}
\author{Alexis Noir-Luhalwe}
\date{\today}

\begin{document}

\maketitle

% ========== ABSTRACT ==========
\begin{abstract}
I develop a hierarchical framework for detecting regime-dependent contagion in high-frequency limit order books motivated by the need to capture distributional stress that correlation-based methods miss. The approach combines robust MAD normalization with temporal Wasserstein distances applied to LOB variables to isolate within-asset distributional ruptures and feeds these signals into a two-level Hidden Markov architecture. Local HMMs extract per-ticker regime probabilities, which a Meta-HMM aggregates to identify sector-wide regimes, with a Direct Global HMM on concatenated features serving as a benchmark. Applied to LOBSTER data for five technology stocks, the framework reveals heterogeneous local regime structures, modest but meaningful global synchronization, and directed causal relationships that point to AAPL as a plausible ``Patient Zero'' of contagion. Lead-lag and event-study evidence indicate that order-flow stress arrives in persistent clusters and triggers coordinated cross-sectional reallocation. Robustness checks against non-parametric clustering and distributional diagnostics support that the regimes reflect genuine market structure rather than modeling artifacts.
\end{abstract}

\textbf{Keywords:} Limit Order Book, Hierarchical Hidden Markov Model, Transfer Entropy, Wasserstein Distance, Regime Detection, Contagion, Market Microstructure, MAD Normalization

% ========== INTRODUCTION ==========
\section{Introduction}
\label{sec:intro}

The increasing fragmentation and algorithmic complexity of modern financial markets have transformed the dynamics of liquidity provision and price discovery. In this context, the study of limit order book (LOB) microstructure has emerged as a critical field, particularly for understanding the mechanisms of contagion and regime-dependent stress propagation across assets. A fundamental challenge in this domain is measuring distributional stress across assets in a way that captures the full richness of non-linear dependencies essential for detecting stress regimes. Traditional correlation-based measures prove inadequate for this task: they only capture linear, first-moment relationships and are highly sensitive to conditional heteroskedasticity, failing to detect the complex distributional changes---shifts in tail behavior, asymmetry, and higher-order moments---that characterize stress propagation in high-frequency markets.

This paper introduces a temporal Wasserstein framework for regime detection and contagion analysis in high-frequency LOB data. Wasserstein distance from optimal transport theory offers a principled solution to measuring distributional stress: unlike correlation matrices that reduce distributional information to a single linear dependence coefficient, Wasserstein distance compares entire probability distributions by computing the minimal cost of transporting one distribution to another. This geometric property enables it to capture non-linear dependencies, tail divergence, and asymmetric stress patterns that correlations overlook. Critically, I adapt Wasserstein distance to a \textit{temporal} setting by comparing each metric's distribution \textit{before} versus \textit{after} each time point, producing a stress signal that spikes precisely at distributional ruptures---a more direct signal for regime transitions than the cross-sectional pairwise approach used in existing work \parencite{horvath2021clustering}. To ensure robustness in noisy microstructure data, I use Median Absolute Deviation (MAD) normalization, a non-parametric approach resilient to the extreme outliers (flash quotes, quote stuffing, transient liquidity vacuums) that destabilize parametric volatility models at sub-second frequencies.

Traditional approaches to regime detection in LOB data suffer from three interrelated limitations that my framework directly addresses. First, univariate HMM models \parencite{slupinski2020hidden, wisebourt2011hierarchical, kth2017online} capture latent states within individual assets but fundamentally overlook the spatial interdependencies between assets that have become crucial in modern fragmented markets where liquidity is distributed across multiple correlated order books. Second, multi-asset LOB models \parencite{kercheval2024attention, zhou2025tlob} introduce advanced architectures to capture spatial dependencies but rely on correlation-based measures that assume static, linear relationships---assumptions invalidated by the non-linear, regime-dependent stress propagation observed during extreme market events \parencite{cont2014price}. Third, standard parametric filtering introduces convergence issues and distributional assumptions that can be problematic at high frequency, where outliers and distributional irregularities are common, blurring genuine regime changes with filtering artifacts.

My Wasserstein-HMM framework addresses these limitations through four methodological steps centered on temporal distributional stress measurement. First, I apply MAD normalization to LOB metrics (micro-price returns, OBI, OFI) to extract robust standardized innovations, eliminating sensitivity to extreme observations without imposing distributional assumptions. Second, I compute \textit{temporal} 1-Wasserstein distances between MAD-normalized distributions before and after each time point, yielding time-varying measures of distributional ruptures per ticker and per metric that capture non-linear dependencies, tail behavior, and asymmetric stress patterns invisible to correlation matrices. The temporal formulation, detects \textit{when} a distributional change occurs within each asset, providing a direct signal for regime transitions that is self-contained per ticker. Third, I construct a two-level HMM architecture: per-ticker local HMMs with individually optimized parameters (MAD window, Wasserstein window, persistence, smoothing) extract posterior state probabilities via the forward-backward algorithm, which a Meta-HMM then aggregates across tickers to detect sector-wide regimes, with a parallel Direct Global HMM trained on concatenated Wasserstein features as benchmark. This hierarchical design resolves label switching, filters noise through a consensus mechanism, and directly detects contagion as coordinated probability shifts. Fourth, I introduce Transfer Entropy to measure directed causal relationships between assets and identify the ``Patient Zero'' of contagion propagation---the asset that initiates regime transitions across the sector.

I apply the framework to LOBSTER data for five technology stocks (AAPL, INTC, GOOG, AMZN, MSFT) on June 21, 2012 to answer a simple question: does one sector behave as one market, or as several micro-markets that occasionally align? The evidence points to a mixed structure. At the single-stock level, each name follows its own regime geometry, with different optimal MAD/Wasserstein windows and persistence; GOOG remains almost entirely in one dominant state (95.1\%), while MSFT rotates across three states (7.5\%/29.3\%/63.2\%). Yet at the sector level, two conceptually different global constructions (Meta-HMM and Direct Global) still agree on about 72\% of states, indicating a common latent signal. This common signal is not constantly shared: local-global synchronization stays low (0--22\%), consistent with limited coupling on a non-crisis day. Directionality analysis then shows that co-movement is organized rather than symmetric: the strongest information transfer is between AAPL and INTC (AAPL$\rightarrow$INTC $=0.00123$ nats, INTC$\rightarrow$AAPL $=0.00113$ nats), and in stress regimes INTC tends to anticipate sector-wide transitions while GOOG lags. The intraday GOOG OFI spike around $t \approx 15{,}000$ ($\sim$11:33 AM), followed by coordinated cross-sectional reallocation, illustrates this mechanism in real time: shocks are absorbed through active rebalancing channels, not passive spillover. Overall, the main takeaway is that sector dynamics are hierarchical---idiosyncratic locally, partially coherent globally, and connected through time-varying directed links.

The remainder of this paper is structured as follows. Section~\ref{sec:lit} reviews the literature with particular attention to the application of Wasserstein distance in financial econometrics. Section~\ref{sec:method} details my methodology, including a comprehensive discussion of why temporal Wasserstein distance is superior to correlation-based measures for detecting distributional stress, how I adapt it to time series through MAD normalization and temporal windows, and why 1-Wasserstein is the appropriate metric for regime detection. Section~\ref{sec:results} presents empirical results. Section~\ref{sec:discuss} discusses implications, and Section~\ref{sec:conclude} concludes.

% ========== LITERATURE REVIEW ==========
\section{Literature Review}
\label{sec:lit}

The analysis of high-frequency market microstructure has undergone a profound transformation with the advent of electronic trading and the fragmentation of liquidity across multiple venues. This evolution has been particularly pronounced in the study of limit order book dynamics, where the granularity of tick-level data has revealed complex patterns of order flow, liquidity provision, and price discovery that challenge traditional market efficiency assumptions. Rather than exhaustively covering all strands of this literature, I focus on four complementary axes that are most relevant to this study: univariate LOB modeling using state-space techniques, multi-asset contagion analysis through spatial dependencies, regime-switching frameworks for capturing non-linear market dynamics, and---most critically for my contribution---the application of Wasserstein distance from optimal transport theory and information-theoretic measures to financial time series.

The foundational work on univariate LOB modeling using Hidden Markov Models established the groundwork for understanding latent market states. \textcite{slupinski2020hidden} demonstrated how HMMs could effectively model liquidity regimes in individual LOBs by capturing the hidden states driving order book imbalances and queue dynamics. This approach was later extended by \textcite{wisebourt2011hierarchical} who introduced hierarchical HMM structures to predict returns based on LOB features, though still confined to single-asset analysis. More recent work by \textcite{kth2017online} implemented non-linear HMM variants for real-time microstructure prediction, maintaining the univariate focus but incorporating more sophisticated state transition dynamics. While these studies successfully captured the temporal evolution of individual LOBs, they fundamentally overlooked the spatial interdependencies that have become crucial in modern fragmented markets where liquidity is distributed across multiple correlated assets.

The emergence of multi-asset LOB analysis addressed some of these limitations by attempting to capture spatial dependencies between different securities. \textcite{kercheval2024attention} introduced attention-based networks to model cross-asset relationships in LOB data, using graph structures to represent the complex web of interactions between different stocks' order books. However, this approach relied on correlation-based similarity measures that only capture linear, first-moment relationships, failing to detect distributional changes in tails, asymmetry, or higher-order moments. \textcite{zhou2025tlob} further developed this line of research with transformer-based architectures, explicitly acknowledging the non-stationarity of spatial correlations as a major limitation of their framework. These correlation-based approaches contrast sharply with the distributional regime-switching behavior observed in real markets, where stress propagation manifests not merely as increased correlation but as fundamental shifts in the joint distribution of order flow and liquidity metrics---shifts that correlation matrices, by construction, cannot detect.

The regime-switching literature provides the theoretical foundation for understanding these dynamic patterns. Pioneering work by \textcite{hamilton1989new} established Markov-switching models as powerful tools for business cycle analysis, while later adaptations by \textcite{agnolucci2019market} and \textcite{zheng2020regime} applied these techniques to financial time series. However, these studies typically focused on aggregated return or volatility measures rather than the granular LOB dynamics that drive high-frequency trading strategies. More critically, they relied on moment-based statistics (means, variances, correlations) to characterize regimes, thereby overlooking the full distributional information that distinguishes genuine stress regimes from periods of elevated but structurally similar volatility.

The Wasserstein distance literature offers a principled solution to measuring distributional divergence that preserves the full richness of probability distributions. Originating in optimal transport theory \parencite{villani2009optimal}, the Wasserstein metric quantifies the minimal cost of transforming one probability distribution into another, providing a geometrically interpretable measure of distributional distance. \textcite{peyre2019computational} provides a comprehensive treatment of the computational aspects, demonstrating efficient algorithms for large-scale applications. \textcite{fournie2019wasserstein} establishes the theoretical foundations for applying Wasserstein distance to financial time series, proving consistency properties and deriving asymptotic distributions for statistical inference. The application of Wasserstein distance to regime detection in financial markets represents a significant methodological advance over correlation-based approaches. \textcite{horvath2021clustering} demonstrate the utility of Wasserstein distance for clustering market regimes, showing that Wasserstein-based regime detection successfully identifies bull markets, bear markets, and crisis periods in equity indices by capturing distributional shifts invisible to correlation matrices. Their key insight is that market regimes differ not merely in mean returns or volatility levels but in the entire shape of return distributions---tail behavior, asymmetry, and higher-order moment structure. Wasserstein distance, by comparing distributions holistically rather than reducing them to correlation coefficients, detects these regime-defining distributional changes. However, their framework has four fundamental limitations for high-frequency microstructure analysis that my work directly addresses.

First, \textcite{horvath2021clustering} apply Wasserstein distance to daily returns aggregated over rolling windows of weeks or months. This temporal resolution---appropriate for asset allocation---cannot capture the intraday contagion dynamics that unfold over seconds to minutes in LOB microstructure. My framework applies Wasserstein distance to LOB metrics computed at 500ms frequency, three orders of magnitude finer than daily data, revealing microstructure phenomena invisible at daily frequency. Second, \textcite{horvath2021clustering} compute Wasserstein distances on raw returns without volatility filtering, implicitly assuming that conditional variance is regime-invariant or irrelevant for regime classification. My framework implements MAD normalization to extract standardized innovations before computing Wasserstein distances, ensuring that measured distributional stress is independent of time-varying local dispersion. Third, \textcite{horvath2021clustering} employ K-means clustering on Wasserstein distance matrices, treating each time period as an independent observation without temporal structure, generating regime assignments that can exhibit spurious transitions. My framework feeds Wasserstein-based stress features into an optimized HMM specification with enforced persistence and post-estimation smoothing, imposing temporal structure consistent with the economic reality that microstructure regimes exhibit substantial inertia. Fourth, \textcite{horvath2021clustering} apply Wasserstein distance in a cross-sectional pairwise manner, comparing distributions \textit{between} assets. My temporal formulation instead compares distributions \textit{before versus after} each time point \textit{within} each asset, directly detecting distributional ruptures rather than cross-asset divergence---a more targeted signal for regime transitions that enables the per-ticker HMM architecture central to my hierarchical framework.

Transfer Entropy \parencite{schreiber2000measuring} extends mutual information to measure directed information flow between time series, providing a non-parametric measure of causal influence. In finance, Transfer Entropy has been applied to identify information leaders in equity markets and to detect contagion pathways during crises. My framework combines Transfer Entropy with regime-based state probabilities rather than raw returns, enabling detection of causal relationships in regime dynamics---who initiates regime transitions rather than who moves prices first.

My framework thus extends the Wasserstein-based regime detection paradigm from macro-level asset allocation to microstructure-level contagion analysis through four methodological steps: sub-second temporal resolution with MAD-normalized innovations, temporal (before-vs-after) rather than cross-sectional (between-asset) Wasserstein formulation, a hierarchical HMM architecture with per-ticker optimization and Meta-HMM aggregation, and Transfer Entropy-based causal analysis for Patient Zero identification.

% ========== METHODOLOGY ==========
\section{Methodology}
\label{sec:method}

My analytical framework integrates robust normalization, temporal Wasserstein distance computation, a hierarchical HMM architecture, and information-theoretic contagion metrics. The pipeline proceeds in five stages: data loading and LOB metric construction, MAD normalization, temporal Wasserstein feature extraction, hierarchical HMM estimation with per-ticker optimization, and contagion analysis via Transfer Entropy and lead-lag dynamics.

\subsection{Data and LOB Metrics}

I utilize LOBSTER (Limit Order Book System -- The Efficient Reconstructor) data \parencite{lobster2019} for five technology stocks (AAPL, INTC, GOOG, AMZN, MSFT) on June 21, 2012 (9:30--16:00 ET). For each asset, I process both order book snapshots (top 5 levels) and individual message events to reconstruct the complete LOB evolution with millisecond precision. Data are synchronized to a common 500ms grid using forward-fill interpolation, yielding approximately 46,400 observations per asset.

I compute three metrics capturing complementary aspects of market dynamics. The micro-price serves as the primary measure of fair value, calculated as a volume-weighted average of the bid and ask prices across the first $n=5$ levels:
\begin{equation}
M_{t,n} = \frac{\sum_{i=1}^{n} (P_{i,t}^{bid} \cdot Q_{i,t}^{ask} + P_{i,t}^{ask} \cdot Q_{i,t}^{bid})}{\sum_{i=1}^{n} (Q_{i,t}^{bid} + Q_{i,t}^{ask})}
\end{equation}
The Order Book Imbalance (OBI) measures the static relative supply-demand pressure within the book:
\begin{equation}
\text{OBI}_{t,n} = \frac{\sum_{i=1}^{n} Q_{i,t}^{bid} - \sum_{i=1}^{n} Q_{i,t}^{ask}}{\sum_{i=1}^{n} Q_{i,t}^{bid} + \sum_{i=1}^{n} Q_{i,t}^{ask}}
\end{equation}
The Multi-Level Order Flow Imbalance (OFI) tracks the dynamic net liquidity flow by aggregating changes in sizes across the first $n=5$ levels:
\begin{equation}
\text{OFI}_{t,n} = \sum_{i=1}^{n} \left( e_{i,t}^{bid} - e_{i,t}^{ask} \right)
\end{equation}
where for each level $i$, the bid-side contribution $e_{i,t}^{bid}$ accounts for price improvements, size changes at constant price, or price retreats. This multi-level approach, extending the framework of \textcite{cont2014price}, provides a more stable representation of net flow by internalizing re-quotes within the top 5 levels of the book.

\subsection{Robust MAD Normalization}
\label{sec:mad}

A critical departure from standard parametric filtering is the adoption of Median Absolute Deviation (MAD) normalization. High-frequency microstructure data exhibit extreme outliers---flash quotes, quote stuffing, transient liquidity vacuums---that can destabilize parametric volatility models. MAD normalization provides a robust, non-parametric tool. For a rolling window of size $w$, I compute the robust z-score:
\begin{equation}
Z_t = \frac{X_t - \widetilde{X}_t}{1.4826 \cdot \text{MAD}_t + \epsilon}
\end{equation}
where $\widetilde{X}_t = \text{median}(X_{t-w}, \ldots, X_t)$ is the rolling median, $\text{MAD}_t = \text{median}(|X_{t-w} - \widetilde{X}_t|, \ldots, |X_t - \widetilde{X}_t|)$ is the rolling median absolute deviation, and $\epsilon = 10^{-9}$ prevents division by zero. The constant $1.4826$ ensures consistency with the standard deviation under normality: for a Gaussian distribution, $\text{MAD} \times 1.4826 = \sigma$.

This approach offers three advantages for high-frequency LOB data. The median is a robust estimator with breakdown point 0.5, meaning up to 50\% of observations can be outliers without affecting the estimate. MAD requires no iterative optimization, it is computed in a single pass. Finally, MAD makes no distributional assumption on the innovation process. The window size $w$ is optimized per ticker (Section~\ref{sec:optimization}), with values ranging from 50 to 150 observations (25--75 seconds at 500ms frequency), allowing each asset to have its own normalization scale appropriate for its microstructure dynamics.

\subsection{Temporal Wasserstein Distance for Distributional Stress Measurement}
\label{sec:wasserstein}

The choice of Wasserstein distance over traditional correlation-based measures, and the adaptation from cross-sectional to temporal formulation, are motivated by fundamental advantages for detecting regime-dependent stress in high-frequency microstructure data. I first establish why correlation-based measures are inadequate, then define the Wasserstein metric and its properties, then describe the temporal adaptation, and finally justify the choice of the 1-Wasserstein metric.

\paragraph{Limitations of Correlation-Based Measures.}

Pearson correlation, the standard approach for measuring cross-sectional dependencies in finance, reduces the relationship between two random variables to a single scalar capturing linear comovement:
\[
\rho_{X,Y} = \frac{\text{Cov}(X,Y)}{\sigma_X \sigma_Y} = \frac{\mathbb{E}[(X - \mu_X)(Y - \mu_Y)]}{\sigma_X \sigma_Y}
\]
This formulation suffers from four critical limitations when applied to high-frequency LOB metrics. First, correlation captures only second-moment comovement, completely discarding information about distributional shape beyond means and variances. In microstructure data, regime changes often manifest as shifts in higher moments---skewness during informed trading episodes, kurtosis spikes during liquidity crises---that correlation cannot detect by construction. If two assets' order flow imbalances exhibit identical means and variances but drastically different tail behavior, correlation would report identical dependence despite fundamentally different stress profiles.

Second, correlation measures linear relationships, imposing the implicit assumption that $Y \approx \beta X + \epsilon$ for some constant $\beta$. However, microstructure dependencies are highly non-linear: order flow toxicity propagates asymmetrically (positive shocks differ from negative shocks), liquidity provision exhibits threshold effects (order book depth collapses discontinuously), and stress contagion follows power-law rather than linear dynamics. Correlation-based regime detection thus conflates genuine distributional regime changes with fluctuations in the linear approximation coefficient, generating spurious transitions.

Third, correlation is undefined when variances approach zero and becomes unstable when they vary substantially over time. High-frequency data exhibit extreme conditional heteroskedasticity: realized volatility can change by an order of magnitude within minutes. Raw correlation computed on such data conflates three distinct phenomena---changes in marginal volatilities, changes in covolatility, and changes in the distributional relationship beyond second moments---making correlation-based regime detection highly unstable.

Fourth, Pearson correlation $\rho_{X_t, Y_t}$ measures contemporaneous comovement, comparing $X$ and $Y$ at the same time index. While lagged correlations can capture lead-lag relationships, they still impose a rigid temporal structure unable to detect distributional shifts where the entire shape of a metric's distribution over a window differs from its distribution over the next window. Wasserstein distance, by contrast, compares empirical distributions aggregated over adjacent windows, capturing how the full distributional profile evolves without imposing point-to-point temporal alignment.

\paragraph{Wasserstein Distance: Definition and Properties.}

The 1-Wasserstein distance between two probability distributions $P$ and $Q$ on $\mathbb{R}$ is defined as:
\[
W_1(P, Q) = \inf_{\gamma \in \Gamma(P,Q)} \mathbb{E}_{(X,Y) \sim \gamma} [|X - Y|] = \int_0^1 |F_P^{-1}(u) - F_Q^{-1}(u)| \, du
\]
where $\Gamma(P,Q)$ is the set of all joint distributions with marginals $P$ and $Q$, and $F_P^{-1}$, $F_Q^{-1}$ are the quantile functions (inverse CDFs). The second equality, valid for univariate distributions, provides a computationally efficient formula: Wasserstein distance equals the $L^1$ distance between quantile functions integrated over the unit interval.

This formulation has four key advantages for measuring distributional stress. It is \textit{distribution-free}: Wasserstein distance makes no parametric assumptions about $P$ or $Q$---no normality, no existence of moments, no linearity. It is well-defined for any probability distributions on $\mathbb{R}$, including those with heavy tails, infinite variance, or multi-modal structure common in microstructure data. It \textit{captures full distributional information}: by comparing quantile functions at all quantile levels $u \in [0,1]$, Wasserstein distance captures differences in location, dispersion, skewness, kurtosis, and tail behavior simultaneously. If $P$ and $Q$ differ in any aspect of their distributional shape, $W_1(P,Q) > 0$ detects this divergence---a pattern correlation misses entirely when differences are confined to higher moments. It is \textit{geometrically interpretable}: $W_1(P,Q)$ quantifies the minimum ``transportation cost'' to transform distribution $P$ into $Q$, providing economic intuition: high Wasserstein distances indicate that distributions are far apart in their support, signaling genuine distributional divergence rather than merely different correlations. Finally, it satisfies the mathematical properties of a \textit{metric}: non-negativity, identity, symmetry, and the triangle inequality, ensuring that Wasserstein-based regime detection has well-defined mathematical foundations where regimes correspond to clusters in the metric space of probability distributions.

\paragraph{From Cross-Sectional to Temporal Wasserstein.}

The existing literature on Wasserstein-based regime detection \parencite{horvath2021clustering} applies Wasserstein distance in a cross-sectional manner, computing $W_1(\hat{P}_{i,t}, \hat{P}_{j,t})$ between assets $i$ and $j$ within a rolling window. While this measures how different two assets' distributions are at time $t$, it conflates two distinct phenomena: genuine within-asset distributional regime changes and persistent cross-asset distributional differences. If asset $i$ has consistently heavier tails than asset $j$, the cross-sectional Wasserstein distance will be persistently high regardless of whether either asset is undergoing a regime transition.

My temporal formulation resolves this issue by comparing the distribution of a single metric \textit{before} and \textit{after} each time point:
\begin{equation}
W_t^{(i,m)} = W_1\!\left(\{Z_{t-w}^{(i,m)}, \ldots, Z_{t}^{(i,m)}\},\, \{Z_{t}^{(i,m)}, \ldots, Z_{t+w}^{(i,m)}\}\right)
\label{eq:temporal_wass}
\end{equation}
where $Z^{(i,m)}$ is the MAD-normalized series for asset $i$ and metric $m$, and $w$ is the window size. This temporal Wasserstein distance $W_t^{(i,m)}$ is small when the distribution is stationary around time $t$ and spikes when a genuine distributional rupture occurs---precisely the signal needed for regime detection. The temporal formulation presents three structural advantages. First, it detects \textit{when} a distributional change occurs within each asset, rather than \textit{how different} two assets are, providing a direct signal for regime transitions. Second, each $W_t^{(i,m)}$ is self-contained per ticker and metric, enabling per-ticker HMM fitting without requiring cross-asset alignment---the foundation of the hierarchical architecture described in Section~\ref{sec:hierarchical}. Third, the temporal Wasserstein series naturally spike during genuine regime transitions, providing a high signal-to-noise input for HMM estimation that is absent in cross-sectional formulations where persistent cross-asset differences obscure transition signals.

The MAD normalization described in Section~\ref{sec:mad} plays a critical role in the temporal formulation. By removing local level and scale effects through median-based centering and scaling, MAD ensures that temporal Wasserstein distances reflect genuine distributional shape changes rather than shifts in location or dispersion that could be attributed to non-stationarity. Without normalization, a gradual drift in the mean of OFI would generate spuriously high temporal Wasserstein distances; with MAD normalization, only changes in the \textit{standardized} distribution---shifts in skewness, kurtosis, tail behavior, or modality---register as distributional ruptures.

\paragraph{Window Size Considerations.}

The window size $w$ controls the temporal resolution of the Wasserstein stress signal. Empirical Wasserstein distance $\hat{W}_1(\hat{P}_n, \hat{Q}_n)$ computed from samples of size $n$ converges to the population Wasserstein distance $W_1(P,Q)$ at rate $O(n^{-1/2})$ under mild regularity conditions \parencite{fournie2019wasserstein}. With $w = 100$ (the maximum in my grid), standard errors are approximately 10\% of the mean distance, providing sufficient precision for regime detection. Smaller windows ($w = 50$, the minimum in my grid) yield noisier estimates but finer temporal resolution, appropriate for assets with rapid microstructure dynamics. I resolve this trade-off through per-ticker optimization (Section~\ref{sec:optimization}), allowing each asset to select the window size that maximizes regime detection quality for its specific dynamics. A 50-second window ($w = 100$ at 500ms frequency) captures microstructure dynamics at trading-relevant horizons where algorithmic execution strategies operate, while a 25-second window ($w = 50$) targets faster dynamics---three orders of magnitude finer than the daily aggregation used in existing Wasserstein regime detection studies.

\paragraph{Why 1-Wasserstein Rather Than Higher Orders.}

The $p$-Wasserstein distance generalizes to $W_p(P,Q) = \left( \int_0^1 |F_P^{-1}(u) - F_Q^{-1}(u)|^p \, du \right)^{1/p}$, with $p=1$ (my choice) versus $p=2$ (Frechet distance) or higher $p$. I select $p=1$ for three reasons. First, my objective is to measure \textit{marginal} distributional stress at each point in time, not to model the joint temporal evolution; the 1-Wasserstein distance compares marginal distributions independently, and the HMM subsequently imposes temporal structure via its transition matrix. Using higher-order metrics that penalize tail discrepancies more heavily would inadvertently overweight extreme observations that may reflect idiosyncratic noise rather than systematic stress. Second, the 1-Wasserstein metric integrates absolute differences linearly, making it less sensitive to extreme quantiles than 2-Wasserstein (which squares differences), providing a balanced treatment of tails versus center appropriate for high-frequency data containing occasional extreme events (flash crashes, quote stuffing). Third, the 1-Wasserstein admits a closed-form solution via sorted quantiles computable in $O(n \log n)$ time, whereas higher-order variants require numerical integration with $O(n^3)$ complexity---a critical advantage given that I compute Wasserstein distances for 5 tickers, 3 metrics, and approximately 46,400 time points. My implementation uses \texttt{scipy.stats.wasserstein\_distance}, with optional Numba JIT acceleration that produces numerically identical results (relative difference $< 10^{-4}$) at substantially reduced computation time.

\subsection{Hierarchical HMM Architecture}
\label{sec:hierarchical}

The core architectural methodology is a two-level HMM that resolves three fundamental problems of flat (single-level) HMMs applied to multi-asset LOB data: label switching, noise filtering, and contagion detection.

At the first level, I fit a Gaussian HMM to the temporal Wasserstein features for each ticker $i$ independently. The observation model is $\mathbf{W}_t^{(i)} | S_t^{(i)} = k \sim \mathcal{N}(\boldsymbol{\mu}_k^{(i)}, \boldsymbol{\Sigma}_k^{(i)})$, where $S_t^{(i)} \in \{0, 1, \ldots, K-1\}$ is the latent regime and $\mathbf{W}_t^{(i)} = (W_t^{(i,\text{Price})}, W_t^{(i,\text{OFI})}, W_t^{(i,\text{OBI})})$ are the three temporal Wasserstein features. The covariance structure is set to diagonal by default, reducing complexity from $O(p^2)$ to $O(p)$ parameters, though I implement an adaptive selection: if the maximum absolute correlation between features exceeds $\tau = 0.70$, I switch to full covariance to avoid model misspecification. After EM estimation with 1000 iterations, I enforce temporal persistence by setting the transition matrix to
\begin{equation}
\mathbf{A}_{jk}^{(i)} = \begin{cases}
\pi^{(i)} & \text{if } j = k \\
\frac{1 - \pi^{(i)}}{K-1} & \text{if } j \neq k
\end{cases}
\end{equation}
where $\pi^{(i)}$ is the per-ticker persistence parameter, and post-estimation smoothing via a majority filter of window $s^{(i)}$ eliminates regime assignments shorter than economically meaningful durations. Critically, in addition to the Viterbi-decoded state labels, I extract the posterior state probabilities via the forward-backward algorithm:
\begin{equation}
\mathbf{p}_t^{(i)} = \left(P(S_t^{(i)} = 0 \mid \mathbf{W}_{1:T}^{(i)}), \ldots, P(S_t^{(i)} = K-1 \mid \mathbf{W}_{1:T}^{(i)})\right)
\end{equation}
These continuous probabilities, rather than discrete labels, serve as input to the Meta-HMM, preserving uncertainty information and resolving label switching: the Meta-HMM learns the semantic mapping between local states by identifying global patterns in the joint probability space regardless of how individual tickers label their regimes.

At the second level, the Meta-HMM observes the concatenated local state probabilities across all $N$ tickers: $\mathbf{X}_t^{\text{meta}} = [\mathbf{p}_t^{(1)}, \ldots, \mathbf{p}_t^{(N)}] \in \mathbb{R}^{N \times K}$. This feature vector captures the full probabilistic state of the sector at each time $t$. The Meta-HMM is itself a Gaussian HMM with $K_G$ global regimes, applying the same persistence and smoothing procedure at the global level with potentially stronger persistence to reflect the greater inertia of sector-wide states. The hierarchical architecture filters noise through a consensus mechanism that only registers global transitions when multiple tickers exhibit coordinated probability shifts, and directly detects contagion as correlated changes in local state probabilities across assets.

As a benchmark, I also fit a Direct Global HMM on the concatenated temporal Wasserstein features across all tickers: $\mathbf{X}_t^{\text{direct}} = [\mathbf{W}_t^{(1)}, \ldots, \mathbf{W}_t^{(N)}] \in \mathbb{R}^{N \times 3}$. This model bypasses the local HMM layer entirely, fitting directly on $N \times 3$-dimensional features. Comparing the two global models reveals whether the two-level hierarchy adds value beyond a flat global approach, with concordance measured via state agreement rate and Adjusted Rand Index.

\subsection{Per-Ticker Parameter Optimization}
\label{sec:optimization}

A distinctive feature of my framework is systematic per-ticker parameter optimization. Rather than imposing a single set of hyperparameters across all assets, I search over a grid of configurations for each ticker independently, varying the MAD window ($w_{\text{MAD}} \in \{50, 100, 150\}$), the Wasserstein window ($w_{\text{Wass}} \in \{50, 100, 150\}$), the local persistence ($\pi \in \{0.85, 0.90, 0.95\}$), and the smoothing window ($s \in \{10, 20, 30\}$), while fixing the number of regimes at $K = 3$, which is a common practice in empirical finance. For each configuration, I evaluate using a composite score:
\begin{equation}
\text{Score} = \text{ARI}(\text{HMM}, \text{K-means}) - \lambda \cdot \text{MMD}_{\text{penalty}}
\end{equation}
where the Adjusted Rand Index measures agreement between HMM regimes and K-means clusters---ensuring the HMM captures genuine structure detectable by non-parametric methods---and the MMD penalty discourages configurations where Maximum Mean Discrepancy between regimes is inconsistent with the expected metric separation. The weight $\lambda = 0.10$ balances exploration and exploitation. This optimization is parallelized across tickers and configurations using \texttt{concurrent.futures}.

\subsection{Contagion Metrics}
\label{sec:contagion}

Transfer Entropy (TE) measures the directed information flow from a source asset to a target asset, quantifying how much knowing the source's past reduces uncertainty about the target's future beyond what the target's own history provides:
\begin{equation}
\text{TE}(X \rightarrow Y) = \sum_{y_{t+1}, y_t, x_t} p(y_{t+1}, y_t, x_t) \log \frac{p(y_{t+1} | y_t, x_t)}{p(y_{t+1} | y_t)}
\end{equation}
where $x_t$ and $y_t$ are discretized versions of the source and target stress probabilities (sum of non-calm regime probabilities). I use $k=2$ lags (1 second) and 10 bins for discretization. The asymmetry $\text{TE}(X \rightarrow Y) - \text{TE}(Y \rightarrow X)$ reveals the net direction of information flow, and I compute a full $N \times N$ TE matrix across all ticker pairs to map the causal network of contagion.

To identify the ``Patient Zero''---the asset that initiates contagion---I combine two complementary metrics: the mean outgoing TE from ticker $i$ to all other tickers, measuring how much causal influence $i$ exerts on the sector, and the leadership score from the Meta-HMM synchronization analysis, measuring how often ticker $i$'s local regime transitions precede global transitions versus follow them. A composite contagion score normalizes both metrics and ranks tickers by their potential to initiate sector-wide regime shifts. I also measure synchronization between local and global regimes through co-transition analysis: the fraction of global regime transitions that coincide with local transitions for each ticker within a $\pm$5 second window.

In addition, I conduct a quantile-conditioned lead-lag analysis to capture state dependence in contagion strength. Specifically, lead-lag correlations are estimated separately within stress quantiles (e.g., $Q10$, $Q50$, $Q90$) of the temporal Wasserstein-based stress signal and then crossed with regime labels. This isolates whether directional dependencies are stable across market conditions or become materially stronger in tail-stress episodes.

\subsection{Robustness Framework}

I validate my framework through four complementary diagnostics. For both local and global models, I compare HMM regime assignments against non-parametric K-means clustering via the Adjusted Rand Index, verifying that identified regimes reflect genuine data structure rather than parametric artifacts. I compute the Maximum Mean Discrepancy (with RBF kernel) between regime-conditional distributions for each metric, verifying that regimes capture genuinely different distributional states---higher MMD values indicate stronger separation between the distributions observed within each regime. I compare Meta-HMM and Direct Global HMM state assignments to assess whether the hierarchical architecture and the flat global model detect similar structures. Finally, I examine the entropy of global state probabilities to assess whether models produce confident or diffuse regime assignments, with near-zero entropy indicating that the model assigns observations to regimes with high certainty.

% ========== RESULTS ==========
\section{Results}
\label{sec:results}

Taken together, the quantile-conditioned lead-lag results, the TE network, and the GOOG event study describe a coherent sector-level story. Transfer Entropy isolates a stable causal core anchored by AAPL and INTC: the strongest directed flows are AAPL$\leftrightarrow$INTC, with AAPL as the net source. Lead-lag analysis then clarifies how this causal core connects to the sector-wide regime: in stressed regimes (R2 at Q90), INTC consistently leads the global regime by roughly 1.5--2 seconds, while GOOG lags the global signal with negative correlation, suggesting a reactive or counter-cyclical role. This aligns with the event study, where a GOOG-specific OFI shock triggers rapid cross-ticker reallocation rather than a sector-wide leading signal. In other words, AAPL and INTC appear to initiate and propagate regime transitions, MSFT and AMZN absorb and transmit these shocks, and GOOG is more often the locus of idiosyncratic liquidity events that are redistributed across the sector. The combined evidence supports a contagion mechanism driven by a small directional core (AAPL/INTC), amplified under stress (Q90), and manifested through reallocation dynamics visible in the GOOG event.

\subsection{Per-Ticker Optimization and Local Regimes}

Table~\ref{tab:local_hmm_params} presents the optimized parameters for each ticker, revealing substantial heterogeneity across assets. AAPL and AMZN favor short MAD and Wasserstein windows (50 observations each), with moderate to high persistence ($\pi = 0.90$ and $0.95$ respectively), suggesting their microstructure dynamics evolve rapidly enough to be captured at relatively local scales. In contrast, GOOG and MSFT require larger windows (MAD $= 100$, Wasserstein $= 150$) and maximal persistence ($\pi = 0.95$), indicating that distributional ruptures in these stocks are more gradual and regime states more persistent. INTC occupies an intermediate position with a short MAD window (50) but longer Wasserstein window (100) and the lowest persistence ($\pi = 0.85$), reflecting a microstructure characterized by rapid local volatility changes but slower distributional shifts. This heterogeneity is economically meaningful: it reflects differences in market-making activity, institutional ownership concentration, and algorithmic trading intensity across stocks, and validates the per-ticker optimization approach---a single configuration applied uniformly would necessarily misfit at least some tickers.

\input{tables/local_hmm_params.tex}

The resulting local regime distributions, presented in Table~\ref{tab:local_regime_distribution}, confirm that regime structure is fundamentally asset-specific. GOOG spends 95.1\% of the trading day in a single dominant regime with only brief excursions into stress states (1.3\% and 3.6\%), consistent with its relatively low intraday volatility on this date. INTC exhibits a similarly concentrated structure (80.6\% in the dominant regime) but with a more substantial secondary regime (13.3\%). At the other extreme, MSFT distributes its time across regimes more evenly (7.5\%/29.3\%/63.2\%), suggesting richer microstructure dynamics with more frequent regime transitions. AAPL falls between these extremes (56.9\%/36.6\%/6.5\%), while AMZN mirrors INTC's concentrated structure with a different secondary allocation (11.4\%).

\input{tables/local_regime_distribution.tex}

Figure~\ref{fig:local_regime_characteristics} illustrates the microstructure characteristics of local regimes for AAPL, showing how the three regimes are differentiated by their temporal Wasserstein feature profiles. The clear separation across regimes---particularly in the OFI dimension---validates that the HMM captures genuinely distinct microstructure states rather than arbitrary partitions of a continuous feature space. Comparable figures for all remaining tickers are presented in Appendix~\ref{app:local_chars}, revealing that each ticker exhibits a unique regime signature consistent with the parameter heterogeneity documented above.

\begin{itemize}
\item \textbf{AAPL.} Regime 0 (56.9\%) is a low-stress baseline; Regime 1 (36.6\%) is an intermediate state with broader increases across Price/OFI/OBI; Regime 2 (6.5\%) concentrates the strongest joint Price--OFI--OBI ruptures, consistent with synchronized microstructure stress.
\item \textbf{INTC.} Regime 0 (80.6\%) is dominant and comparatively calm; Regime 1 (13.3\%) is imbalance-dominant (OBI peak); Regime 2 (6.1\%) is flow-price dominant (OFI and Price peaks), indicating two distinct stress channels.
\item \textbf{GOOG.} Regime 0 (95.1\%) reflects persistent baseline conditions; the rare Regime 1 (1.3\%) is OFI-shock dominated; Regime 2 (3.6\%) is OBI-surge dominated, separating flow toxicity from book-skew episodes.
\item \textbf{AMZN.} Regime 0 (83.9\%) is the low-stress background; Regime 2 (11.4\%) captures strong Price--OFI ruptures with more moderate imbalance; the rare Regime 1 (4.7\%) concentrates the strongest joint OFI--OBI stress.
\item \textbf{MSFT.} Regime 2 (63.2\%) behaves as the baseline low Price/OFI state; Regime 1 (29.3\%) is OFI-led activity with limited price dislocation; Regime 0 (7.5\%) concentrates the strongest Price--OBI disturbances.
\end{itemize}

\begin{figure}[H]
\centering
\includegraphics[width=0.95\textwidth]{figures/regime_characteristics_local_AAPL.png}
\caption{Local regime characteristics for AAPL. Error bars represent $\pm 1$ standard deviation. The separation across regimes, particularly for OFI, validates that the HMM captures genuinely distinct microstructure states. Comparable figures for the remaining four tickers are presented in Appendix~\ref{app:local_chars}.}
\label{fig:local_regime_characteristics}
\end{figure}

\subsection{Global Regimes and Synchronization}

The Meta-HMM and Direct Global HMM produce global regime assignments with approximately 72.1\% state agreement (Table~\ref{tab:sync_global_comparison}), indicating that both approaches detect broadly similar sector-wide patterns despite fundamentally different input representations---local state probabilities for the Meta-HMM versus raw Wasserstein features for the Direct model. This substantial but imperfect concordance suggests that the two models capture complementary aspects of sector dynamics: the Meta-HMM, operating on already-filtered local state probabilities, achieves near-deterministic regime assignments (mean entropy $= 0.000$, Table~\ref{tab:entropy_global}), while the Direct model, which must extract regime structure directly from raw Wasserstein features, exhibits slightly higher uncertainty (entropy $= 0.002$). This contrast reflects the information compression inherent in the hierarchical architecture: local HMMs reduce the high-dimensional Wasserstein feature space to a low-dimensional probability simplex, and the Meta-HMM operates on this compressed representation with correspondingly higher confidence.

\input{tables/sync_global_comparison.tex}

\input{tables/entropy_global.tex}

Table~\ref{tab:local_global_sync} presents the synchronization metrics between local and global regimes, revealing the coupling strength between each ticker and the sector-wide dynamics. GOOG exhibits the highest synchronization rate (22.0\%) with 13 co-transitions out of 59 global transitions, but paradoxically shows a leadership score of $-1.0$, indicating that it consistently \textit{lags} rather than leads global transitions---its local regime responds to sector-wide shifts rather than initiating them. AMZN shows a moderate synchronization rate (8.5\%) with a positive leadership score ($+0.09$), suggesting it slightly anticipates sector-wide regime shifts. INTC and AAPL display zero co-transition synchronization despite having nonzero leadership scores, indicating that their local transitions occur frequently but never coincide with global transition windows---their regime dynamics are largely decoupled from the sector-wide pattern. MSFT shows minimal synchronization (3.4\%) with a negative leadership score ($-0.43$), lagging global transitions. These weak synchronization rates across the board are consistent with moderate intra-sector dependence on a non-crisis day---a baseline against which crisis-day synchronization can be measured, where one would expect substantially higher coordination as common shocks propagate simultaneously.

\input{tables/local_global_sync.tex}

Figure~\ref{fig:temporal_comparison_aapl} compares the local, meta-global, and direct-global regime timelines for AAPL, illustrating the relationship between the three levels of regime detection. The partial alignment between local and global timelines reflects the moderate coupling between AAPL's microstructure dynamics and sector-wide patterns, while the discrepancies highlight periods where AAPL's idiosyncratic dynamics diverge from the sector.

\begin{figure}[H]
\centering
\includegraphics[width=0.95\textwidth]{figures/temporal_regime_comparison_AAPL.png}
\caption{Temporal regime comparison for AAPL. Top: local HMM regimes. Middle: Meta-HMM global regimes. Bottom: Direct Global HMM regimes. Comparable figures for all tickers are in Appendix~\ref{app:temporal_comparisons}.}
\label{fig:temporal_comparison_aapl}
\end{figure}

\subsection{Stress Decomposition and Lead-Lag Dynamics}

Figure~\ref{fig:stress_decomp_meta} presents the ticker-specific stress decomposition with Meta-HMM regime overlays, providing a visual validation of the regime identification. Periods classified as high-stress regimes by the Meta-HMM correspond to visible elevations in the temporal Wasserstein distances across multiple tickers simultaneously, particularly in the OFI dimension. The clear association between colored regime backgrounds and stress level patterns confirms that the Meta-HMM identifies economically meaningful periods of coordinated distributional change rather than arbitrary partitions.

\begin{figure}[H]
\centering
\includegraphics[width=\textwidth]{figures/stress_decomposition_meta.png}
\caption{Stress decomposition by ticker and metric with Meta-HMM regime overlay. Top: Price stress. Middle: OBI stress. Bottom: OFI stress. Colored backgrounds indicate Meta-HMM global regimes. The association between regime transitions and stress spikes validates the regime identification.}
\label{fig:stress_decomp_meta}
\end{figure}

The cross-ticker lead-lag relationships (Table~\ref{tab:leadlag_between_tickers}) identify pairs with significant temporal dependencies in their Wasserstein stress dynamics. The INTC--MSFT pair exhibits the strongest correlation (0.253 at $+10$ seconds lag), indicating that INTC's temporal Wasserstein stress is a strong predictor of MSFT stress 10 seconds later---consistent with the close relationship between these two stocks as technology sector components with overlapping institutional investor bases. The AAPL--AMZN pair shows the second strongest relationship (0.222 at $+7$ seconds), while AAPL--MSFT and AAPL--INTC display significant correlations at negative lags, indicating that these pairs follow rather than lead AAPL's dynamics. Notably, AAPL--GOOG ($r = -0.010$) and INTC--AMZN ($r = 0.007$, $p = 0.154$) show near-zero and statistically insignificant correlations, suggesting that these pairs exhibit largely independent microstructure dynamics despite belonging to the same sector---a finding that underscores the heterogeneity of intra-sector dependencies.

\input{tables/leadlag_between_tickers_top.tex}

Figure~\ref{fig:leadlag_grid} presents the full $3 \times 3$ multi-metric lead-lag analysis across stress quantiles, revealing two key patterns. OFI exhibits strong autocorrelation ($\rho > 0.8$ for $\pm$5 seconds), indicating that order flow toxicity arrives in persistent clusters rather than randomly, consistent with informed trading strategies that execute over multi-second horizons where large orders are sliced into smaller sequences to minimize market impact. Lead-lag structure intensifies systematically during high-stress periods: Q90 correlations substantially exceed Q10 across all metric pairs, demonstrating that microstructure dependencies amplify precisely when market conditions are most stressed---a non-linear pattern that correlation-based measures would attribute to heteroskedasticity rather than genuine distributional regime change, validating the Wasserstein-based approach.

\begin{figure}[H]
\centering
\includegraphics[width=\textwidth]{figures/leadlag_multimetric_grid.png}
\caption{Multi-metric lead-lag analysis by stress quantile. Each panel displays cross-correlations between source and target metrics across stress quantiles (Q10 $=$ calm, Q50 $=$ normal, Q90 $=$ stress). Negative lags indicate source precedes target. Key findings: OFI exhibits strong autocorrelation, and lead-lag structure intensifies during high-stress periods (Q90 correlations substantially exceed Q10).}
\label{fig:leadlag_grid}
\end{figure}

\subsection{Transfer Entropy and Patient Zero}

Table~\ref{tab:transfer_entropy} presents the top directed information flow relationships between tickers, computed from regime state probabilities. The AAPL$\rightarrow$INTC pair exhibits the strongest Transfer Entropy (0.000771 nats), followed by the reverse direction INTC$\rightarrow$AAPL (0.000588 nats). This bidirectional but asymmetric flow indicates that AAPL exerts greater causal influence on INTC than vice versa, consistent with AAPL's larger market capitalization and more active institutional trading. The asymmetry $\text{TE}(\text{AAPL} \rightarrow \text{INTC}) - \text{TE}(\text{INTC} \rightarrow \text{AAPL}) = 0.000183$ nats confirms the net causal direction from AAPL to INTC. The AMZN$\rightarrow$AAPL connection (0.000498 nats) represents the third strongest relationship, revealing that AMZN's regime dynamics contain predictive information about AAPL's future state transitions. GOOG appears primarily as a receiver rather than a sender of information, with GOOG$\rightarrow$AAPL (0.000264 nats) being its strongest outgoing link---consistent with GOOG's high synchronization rate but negative leadership score documented in the synchronization analysis.

\input{tables/transfer_entropy_top.tex}

The Patient Zero analysis combines outgoing Transfer Entropy with the leadership score from synchronization analysis. AAPL emerges as the strongest candidate for contagion initiator: it has the highest total outgoing TE (summing AAPL$\rightarrow$INTC, AAPL$\rightarrow$AMZN, and AAPL$\rightarrow$MSFT) and, despite a negative leadership score in the synchronization analysis, its causal influence on other tickers' regime dynamics is unambiguous from the TE matrix. INTC serves as a secondary transmission node, with strong bidirectional connections to AAPL and moderate outgoing TE to the rest of the sector. This causal structure---AAPL as initiator, INTC as amplifier, GOOG/AMZN/MSFT as receivers---provides a directed contagion network that cannot be inferred from symmetric correlation matrices, demonstrating the value of Transfer Entropy for identifying information leaders in regime dynamics.

\subsection{Event Study: GOOG OFI Spike}

Figure~\ref{fig:event_study} highlights a representative liquidity episode in which GOOG exhibits a large OFI spike at $t \approx 15{,}000$ (approximately 11:33 AM), followed by synchronous OFI moves in the other four stocks. The 33-minute context panel situates the event within broader intraday dynamics, while the 4-minute zoom makes the cross-ticker co-movement explicit. GOOG shows a large negative OFI change ($\Delta \text{OFI} \approx -55$), whereas MSFT, AMZN, AAPL, and INTC display positive OFI changes totaling approximately $+450$. One plausible interpretation is coordinated cross-sectional reallocation toward close substitutes after a localized GOOG liquidity shock. However, this remains an inference rather than a causal proof: alternative explanations include common latent shocks, quote-update synchronization, or shared algorithmic responses across names. The event is therefore best viewed as suggestive evidence consistent with active reallocation channels, not as a universal mechanism.

\begin{figure}[H]
\centering
\includegraphics[width=\textwidth]{figures/event_study_goog_spike_optimal.png}
\caption{Event study: GOOG OFI spike and cross-ticker co-movements. Top: 33-minute context showing the spike at $t \approx 15{,}000$ ($\sim$11:33 AM). Bottom: 4-minute zoom revealing synchronized cross-ticker OFI responses. GOOG exhibits a large negative OFI change while MSFT, AMZN, AAPL, and INTC show positive OFI responses; this pattern is consistent with coordinated reallocation but does not, by itself, establish causality.}
\label{fig:event_study}
\end{figure}

\subsection{Robustness Diagnostics}

The ARI diagnostics for global models (Table~\ref{tab:ari_global}) reveal that the Meta-HMM shows higher agreement with K-means (ARI $= 0.174$) than the Direct model (ARI $= 0.129$), suggesting the hierarchical approach captures more of the non-parametric cluster structure inherent in the data. The meta-vs-direct ARI of 0.219 indicates fair agreement between the two global approaches, confirming that they detect related but not identical structures---consistent with the $\approx$72\% state agreement documented in Table~\ref{tab:sync_global_comparison}. These moderate ARI values are expected: the HMM imposes temporal persistence that K-means lacks, so perfect agreement would indicate that temporal structure adds no information, while zero agreement would suggest the HMM captures parametric artifacts rather than genuine distributional clusters.

\input{tables/ari_global_comparisons.tex}

The Maximum Mean Discrepancy between regimes at the global level (Table~\ref{tab:mmd_global}) confirms that regimes capture genuinely distinct distributional states. The Meta-HMM shows effective separation on both OBI and OFI, with a clear advantage on OBI (MMD $= 0.521$ vs. $0.292$ for the Direct model) and still substantial OFI separation ($0.710$). The Direct model reaches its strongest separation through OFI ($0.746$ vs. $0.710$ for Meta-HMM), indicating that its regime discrimination is more concentrated in that channel. Price separation is moderate in both models ($0.158$ and $0.180$), consistent with the finding that price dynamics are less regime-dependent than microstructure metrics. Detailed local-level MMD diagnostics, ARI comparisons between Meta-HMM and local models, and additional lead-lag tables are presented in Appendix~\ref{app:robustness}.

\input{tables/mmd_global.tex}

% ========== DISCUSSION ==========
\section{Discussion}
\label{sec:discuss}

\subsection{Methodological Contributions}

My analysis makes four methodological contributions to the market microstructure literature. First, I demonstrate that MAD normalization provides a robust, non-parametric alternative to standard parametric filtering for high-frequency LOB data. The rolling median-based approach eliminates sensitivity to outliers without imposing distributional assumptions, avoids convergence failures common with iterative parametric models at sub-second frequencies, and introduces a tunable window parameter that can be optimized per ticker. This represents a practical advance for production trading systems where robustness to extreme market conditions is paramount.

Second, the temporal Wasserstein formulation---comparing before-vs-after distributions rather than cross-sectional pairwise distances---provides a more direct signal for regime transition detection. By measuring distributional ruptures within each ticker independently, the temporal approach yields per-ticker features that naturally spike during genuine regime changes, improving HMM estimation quality. The temporal formulation also resolves the confound inherent in cross-sectional Wasserstein, where persistent cross-asset distributional differences obscure transition signals. This separation of concerns---Wasserstein for temporal distributional stress, HMM for regime structure---provides methodological clarity absent from existing approaches.

Third, the hierarchical HMM architecture resolves the label switching problem inherent in multi-asset regime detection, filters noise through a consensus mechanism, and directly detects contagion as coordinated probability shifts across assets. The comparison between the Meta-HMM and the Direct Global HMM reveals that both approaches detect broadly similar structures ($\approx$72\% state agreement), but with complementary regime characterizations: the Meta-HMM separates regimes effectively across both OBI and OFI (with its strongest edge on OBI), whereas the Direct model's discrimination is more strongly concentrated in OFI (Table~\ref{tab:mmd_global}). This complementarity suggests that neither approach alone captures the full richness of sector-wide regime dynamics, and that combining their signals could improve regime detection accuracy.

Fourth, the integration of Transfer Entropy with regime-based state probabilities enables detection of directed causal relationships in regime dynamics. Unlike traditional lead-lag analysis based on returns or correlations, TE on regime probabilities measures who initiates regime transitions, providing a principled framework for Patient Zero identification that reveals asymmetric causal structure invisible to symmetric correlation matrices.

\subsection{Economic Implications}

The empirical results on June 21, 2012 reveal several important patterns. The heterogeneity of local regime structures across tickers is a key finding: rather than a single regime structure characterizing the technology sector, each asset exhibits its own microstructure signature with different optimal parameters. GOOG spends 95\% of the day in a single regime while MSFT distributes time across three regimes more evenly. This heterogeneity reflects differences in market-making activity, institutional ownership, and algorithmic trading intensity, and validates the per-ticker optimization approach that constitutes a key design principle of the hierarchical architecture.

The weak but significant global synchronization is consistent with expectations for a non-crisis day. The moderate ARI values (0.129--0.219) and low synchronization rates (0--22\%) indicate that cross-asset regime alignment is not dominant and idiosyncratic microstructure dynamics prevail. On a crisis day, one would expect substantially higher synchronization as common shocks force coordinated regime transitions across assets. The framework thus provides a natural baseline for measuring the intensity of sector-wide contagion by comparing synchronization rates across market conditions.

Transfer Entropy reveals asymmetric information flow, with the AAPL--INTC pair exhibiting the strongest bidirectional connection, consistent with the historical relationship between these stocks as technology sector components with overlapping institutional investor bases. The directed nature of TE reveals that AAPL initiates and INTC amplifies contagion---a causal structure that cannot be inferred from lead-lag correlations alone.

\subsection{Practical Applications}

The GOOG event study confirms that stress-driven reallocation follows a hierarchical absorption pattern, with more liquid alternatives absorbing the bulk of capital flows. This demonstrates active portfolio rebalancing rather than passive contagion---an important distinction for regulatory assessments of systemic risk in high-frequency markets. The stabilizing nature of this redistribution argues against overly restrictive HFT regulations: stress is absorbed via reallocation rather than amplification, and the near-zero-sum character of the flows indicates that liquidity migrates rather than evaporates.

From a practical standpoint, the framework enables real-time monitoring of cross-sectional stress and provides actionable intelligence through Patient Zero identification for risk managers seeking leading indicators of sector-wide regime shifts. The per-ticker optimization ensures that regime detection is calibrated to each asset's microstructure, avoiding the false positives that arise when a single configuration is applied to heterogeneous assets.

Several limitations should be noted. My analysis covers a single trading day; extending to multiple days, including crisis periods, would test the framework's ability to detect varying contagion intensities and validate the synchronization baseline. The parameter optimization uses a fixed grid rather than adaptive or Bayesian search, which could improve parameter selection efficiency. The Transfer Entropy computation relies on histogram-based discretization, introducing sensitivity to bin count; continuous TE estimators (e.g., KSG estimator) could improve robustness. Finally, the framework currently focuses on within-sector contagion; cross-sector analysis would provide a more complete picture of systemic risk propagation.

% ========== CONCLUSION ==========
\section{Conclusion}
\label{sec:conclude}

This paper introduces a hierarchical regime-detection and contagion framework for high-frequency limit order books that combines four methodological innovations: robust MAD normalization, temporal Wasserstein distances, a two-level HMM architecture (local per-ticker and global Meta-HMM), and Transfer Entropy-based causal analysis. The framework addresses fundamental limitations of existing approaches---outlier sensitivity of parametric filtering, label switching in multi-asset HMMs, the confounding of transition signals by persistent cross-asset differences in cross-sectional Wasserstein formulations, and the inability of correlation-based measures to detect directed causal relationships in regime dynamics.

Empirically, I identify heterogeneous local regime structures across five technology stocks, with per-ticker parameter optimization revealing that no single configuration adequately captures the diverse microstructure signatures within a sector. Global synchronization is weak but statistically significant, consistent with moderate intra-sector dependence on a non-crisis day and providing a natural baseline for crisis-day comparison. Transfer Entropy reveals directed causal relationships and enables Patient Zero identification, pointing to AAPL as the primary contagion initiator and INTC as a secondary amplifier. Lead-lag analysis documents regime-dependent predictability with INTC leading global dynamics by 10 seconds, and an event study of a GOOG liquidity event confirms hierarchical capital reallocation demonstrating active portfolio rebalancing rather than passive spillover.

Robustness validation through ARI comparisons (meta vs. K-means: 0.174), MMD diagnostics, meta-vs-direct concordance ($\approx$72\%), and entropy analysis confirms that identified regimes represent genuine market structures. The framework provides a principled foundation for regime-aware microstructure analysis, combining the distributional sensitivity of Wasserstein distance, the temporal structure of HMMs, and the causal directionality of Transfer Entropy into an integrated pipeline applicable to real-time monitoring of cross-sectional stress and portfolio risk management.

\printbibliography

% ========== APPENDIX ==========
\appendix
\section{Appendix}

\subsection{Local Regime Characteristics}
\label{app:local_chars}

\begin{figure}[H]
\centering
\begin{subfigure}[b]{0.48\textwidth}
\includegraphics[width=\textwidth]{figures/regime_characteristics_local_INTC.png}
\caption{INTC}
\end{subfigure}
\hfill
\begin{subfigure}[b]{0.48\textwidth}
\includegraphics[width=\textwidth]{figures/regime_characteristics_local_GOOG.png}
\caption{GOOG}
\end{subfigure}
\\[0.5em]
\begin{subfigure}[b]{0.48\textwidth}
\includegraphics[width=\textwidth]{figures/regime_characteristics_local_AMZN.png}
\caption{AMZN}
\end{subfigure}
\hfill
\begin{subfigure}[b]{0.48\textwidth}
\includegraphics[width=\textwidth]{figures/regime_characteristics_local_MSFT.png}
\caption{MSFT}
\end{subfigure}
\caption{Local regime characteristics for INTC, GOOG, AMZN, and MSFT. Error bars represent $\pm 1$ standard deviation. Each ticker exhibits distinct regime signatures, confirming the value of per-ticker parameter optimization.}
\label{fig:local_regime_chars_appendix}
\end{figure}

\subsection{Local HMM Timelines and Histograms}
\label{app:local_timelines}

\begin{figure}[H]
\centering
\begin{subfigure}[b]{0.48\textwidth}
\includegraphics[width=\textwidth]{figures/hmm_local_AAPL_timeline.png}
\caption{AAPL}
\end{subfigure}
\hfill
\begin{subfigure}[b]{0.48\textwidth}
\includegraphics[width=\textwidth]{figures/hmm_local_INTC_timeline.png}
\caption{INTC}
\end{subfigure}
\\[0.5em]
\begin{subfigure}[b]{0.48\textwidth}
\includegraphics[width=\textwidth]{figures/hmm_local_GOOG_timeline.png}
\caption{GOOG}
\end{subfigure}
\hfill
\begin{subfigure}[b]{0.48\textwidth}
\includegraphics[width=\textwidth]{figures/hmm_local_AMZN_timeline.png}
\caption{AMZN}
\end{subfigure}
\\[0.5em]
\begin{subfigure}[b]{0.48\textwidth}
\includegraphics[width=\textwidth]{figures/hmm_local_MSFT_timeline.png}
\caption{MSFT}
\end{subfigure}
\caption{Local HMM regime timelines for all five tickers throughout the trading day.}
\label{fig:local_timelines}
\end{figure}

\begin{figure}[H]
\centering
\begin{subfigure}[b]{0.48\textwidth}
\includegraphics[width=\textwidth]{figures/hmm_local_AAPL_hist.png}
\caption{AAPL}
\end{subfigure}
\hfill
\begin{subfigure}[b]{0.48\textwidth}
\includegraphics[width=\textwidth]{figures/hmm_local_INTC_hist.png}
\caption{INTC}
\end{subfigure}
\\[0.5em]
\begin{subfigure}[b]{0.48\textwidth}
\includegraphics[width=\textwidth]{figures/hmm_local_GOOG_hist.png}
\caption{GOOG}
\end{subfigure}
\hfill
\begin{subfigure}[b]{0.48\textwidth}
\includegraphics[width=\textwidth]{figures/hmm_local_AMZN_hist.png}
\caption{AMZN}
\end{subfigure}
\\[0.5em]
\begin{subfigure}[b]{0.48\textwidth}
\includegraphics[width=\textwidth]{figures/hmm_local_MSFT_hist.png}
\caption{MSFT}
\end{subfigure}
\caption{Local HMM regime histograms for all five tickers.}
\label{fig:local_histograms}
\end{figure}

\subsection{Temporal Regime Comparisons (All Tickers)}
\label{app:temporal_comparisons}

\begin{figure}[H]
\centering
\begin{subfigure}[b]{0.48\textwidth}
\includegraphics[width=\textwidth]{figures/temporal_regime_comparison_INTC.png}
\caption{INTC}
\end{subfigure}
\hfill
\begin{subfigure}[b]{0.48\textwidth}
\includegraphics[width=\textwidth]{figures/temporal_regime_comparison_GOOG.png}
\caption{GOOG}
\end{subfigure}
\\[0.5em]
\begin{subfigure}[b]{0.48\textwidth}
\includegraphics[width=\textwidth]{figures/temporal_regime_comparison_AMZN.png}
\caption{AMZN}
\end{subfigure}
\hfill
\begin{subfigure}[b]{0.48\textwidth}
\includegraphics[width=\textwidth]{figures/temporal_regime_comparison_MSFT.png}
\caption{MSFT}
\end{subfigure}
\caption{Temporal regime comparisons (local vs. Meta-HMM vs. Direct Global) for all tickers except AAPL (shown in Figure~\ref{fig:temporal_comparison_aapl}).}
\label{fig:temporal_comparisons_appendix}
\end{figure}

\subsection{Global HMM Timelines and Characteristics}
\label{app:global}

\begin{figure}[H]
\centering
\begin{subfigure}[b]{0.48\textwidth}
\includegraphics[width=\textwidth]{figures/hmm_meta_timeline.png}
\caption{Meta-HMM timeline}
\end{subfigure}
\hfill
\begin{subfigure}[b]{0.48\textwidth}
\includegraphics[width=\textwidth]{figures/hmm_direct_timeline.png}
\caption{Direct Global HMM timeline}
\end{subfigure}
\\[0.5em]
\begin{subfigure}[b]{0.48\textwidth}
\includegraphics[width=\textwidth]{figures/hmm_meta_hist.png}
\caption{Meta-HMM histogram}
\end{subfigure}
\hfill
\begin{subfigure}[b]{0.48\textwidth}
\includegraphics[width=\textwidth]{figures/hmm_direct_hist.png}
\caption{Direct Global HMM histogram}
\end{subfigure}
\caption{Global HMM regime timelines and histograms for both the Meta-HMM and Direct Global models.}
\label{fig:global_timelines}
\end{figure}

\begin{figure}[H]
\centering
\begin{subfigure}[b]{0.48\textwidth}
\includegraphics[width=\textwidth]{figures/regime_characteristics_meta.png}
\caption{Meta-HMM regime characteristics}
\end{subfigure}
\hfill
\begin{subfigure}[b]{0.48\textwidth}
\includegraphics[width=\textwidth]{figures/regime_characteristics_global_direct.png}
\caption{Direct Global HMM regime characteristics}
\end{subfigure}
\caption{Global regime characteristics for both models.}
\label{fig:global_regime_chars}
\end{figure}

\subsection{Synchronization and Entropy}
\label{app:sync}

\begin{figure}[H]
\centering
\begin{subfigure}[b]{0.48\textwidth}
\includegraphics[width=\textwidth]{figures/sync_local_global.png}
\caption{Local$\rightarrow$Global synchronization rates}
\end{subfigure}
\hfill
\begin{subfigure}[b]{0.48\textwidth}
\includegraphics[width=\textwidth]{figures/entropy_global.png}
\caption{Global entropy distribution (Meta vs. Direct)}
\end{subfigure}
\caption{Left: synchronization rates between local and global regime transitions for each ticker. Right: entropy distributions comparing regime assignment confidence of the Meta-HMM and Direct Global models.}
\label{fig:sync_entropy}
\end{figure}

\subsection{Robustness Diagnostics}
\label{app:robustness}

\input{tables/ari_meta_vs_local.tex}

\input{tables/mmd_local.tex}

\subsection{Stress Decomposition (Direct Global)}
\label{app:stress_direct}

\begin{figure}[H]
\centering
\includegraphics[width=\textwidth]{figures/stress_decomposition_direct.png}
\caption{Stress decomposition with Direct Global HMM overlay. Compare with Figure~\ref{fig:stress_decomp_meta} (Meta-HMM overlay) to observe differences in regime boundary placement.}
\label{fig:stress_decomp_direct}
\end{figure}

\subsection{Local HMM Feature Distributions}
\label{app:local_features}

\begin{figure}[H]
\centering
\begin{subfigure}[b]{0.48\textwidth}
\includegraphics[width=\textwidth]{figures/hmm_local_AAPL_features.png}
\caption{AAPL}
\end{subfigure}
\hfill
\begin{subfigure}[b]{0.48\textwidth}
\includegraphics[width=\textwidth]{figures/hmm_local_INTC_features.png}
\caption{INTC}
\end{subfigure}
\\[0.5em]
\begin{subfigure}[b]{0.48\textwidth}
\includegraphics[width=\textwidth]{figures/hmm_local_GOOG_features.png}
\caption{GOOG}
\end{subfigure}
\hfill
\begin{subfigure}[b]{0.48\textwidth}
\includegraphics[width=\textwidth]{figures/hmm_local_AMZN_features.png}
\caption{AMZN}
\end{subfigure}
\\[0.5em]
\begin{subfigure}[b]{0.48\textwidth}
\includegraphics[width=\textwidth]{figures/hmm_local_MSFT_features.png}
\caption{MSFT}
\end{subfigure}
\caption{Local HMM feature distributions by regime for all tickers.}
\label{fig:local_features_appendix}
\end{figure}

\subsection{Global HMM Feature Distributions}
\label{app:global_features}

\begin{figure}[H]
\centering
\begin{subfigure}[b]{0.48\textwidth}
\includegraphics[width=\textwidth]{figures/hmm_meta_features.png}
\caption{Meta-HMM features by regime}
\end{subfigure}
\hfill
\begin{subfigure}[b]{0.48\textwidth}
\includegraphics[width=\textwidth]{figures/hmm_direct_features.png}
\caption{Direct Global HMM features by regime}
\end{subfigure}
\caption{Global HMM feature distributions by regime.}
\label{fig:global_features_appendix}
\end{figure}

\subsection{Additional Lead-Lag Analysis}
\label{app:leadlag}

\input{tables/leadlag_local_global_top.tex}

\input{tables/leadlag_significant_global_top10.tex}

\begin{figure}[H]
\centering
\begin{subfigure}[b]{0.48\textwidth}
\includegraphics[width=\textwidth]{figures/leadlag_local_AAPL_OFI_OFI_R0.png}
\caption{AAPL: OFI$\rightarrow$OFI (Local, R0)}
\end{subfigure}
\hfill
\begin{subfigure}[b]{0.48\textwidth}
\includegraphics[width=\textwidth]{figures/leadlag_local_GOOG_OFI_OFI_R0.png}
\caption{GOOG: OFI$\rightarrow$OFI (Local, R0)}
\end{subfigure}
\\[0.5em]
\begin{subfigure}[b]{0.48\textwidth}
\includegraphics[width=\textwidth]{figures/leadlag_meta_AAPL_OFI_OFI_R0.png}
\caption{AAPL: OFI$\rightarrow$OFI (Meta, R0)}
\end{subfigure}
\hfill
\begin{subfigure}[b]{0.48\textwidth}
\includegraphics[width=\textwidth]{figures/leadlag_meta_INTC_OBI_OBI_R0.png}
\caption{INTC: OBI$\rightarrow$OBI (Meta, R0)}
\end{subfigure}
\caption{Selected lead-lag plots showing significant autocorrelation patterns.}
\label{fig:leadlag_selected}
\end{figure}

\subsection{Implementation Details}
\label{app:implementation}

The complete analysis pipeline is implemented in Python using \texttt{polars} (v1.0+) for high-performance data loading via LazyFrame evaluation, \texttt{scipy.stats} for Wasserstein distance computation and statistical tests, \texttt{hmmlearn} (v0.3) for Gaussian HMM estimation with the forward-backward algorithm, \texttt{scikit-learn} (v1.3) for K-means clustering and ARI/Silhouette analysis, \texttt{numba} for optional JIT-compiled Wasserstein acceleration, \texttt{networkx} for contagion network visualization, and \texttt{concurrent.futures} for parallelized per-ticker optimization.

Local HMMs are fitted with 3 states, diagonal covariance (adaptive to full if $|\rho| \geq 0.70$), 1000 EM iterations, and per-ticker optimized persistence and smoothing. The Meta-HMM uses 3 global regimes with diagonal covariance, while the Direct Global HMM operates on the concatenated 15-dimensional Wasserstein feature space. Transfer Entropy is estimated via histogram-based discretization with $k=2$ lags and 10 bins. MMD uses an RBF kernel with median heuristic bandwidth, and K-means employs 10 initializations with StandardScaler preprocessing.

\end{document}
